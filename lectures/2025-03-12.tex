%!TEX root = ../notes.tex
\section{March 12, 2025}
\label{20250312}

This lecture we cover in more detail Succinct Non-Interactive Arguments (SNARGs).

\subsection{Succinct Non-Interactive Argument (SNARG)}
This brings us to succinct arguments, which are seemingly not quite possible.
\begin{definition}[Succinct Non-Interactive Arguments]
    A non-interactive proof/argument system is \ul{succinct} if
    \begin{itemize}
        \item The proof $\pi$ is of length $|\pi| = \mathrm{poly}(\lambda, \log |c|)$.
        \item The verifier runs in time $\mathrm{poly}(\lambda, |x|, \log|c|)$.
    \end{itemize}
\end{definition}
Additionally, SNARKs are Succinct Non-Interactive Arguments of Knowledge. A zk-SNARG or zk-SNARK additionally guarantees zero-knowledge property.

\emph{Why succinct proofs?} Here are some examples where we might want succinct proofs.
\begin{example}[Verifiable Computation]
    The client sends some $x$ to the server, along with function $f$. The server sends back $y = f(x)$ and a proof. The client wants to check if the computation was done correctly.

    \pseudocodeblock{
        \textbf{Server} \< \< \textbf{Client}\\
            \< \sendmessageleft*{x} \< \\
            \< \sendmessageleft*{\text{compute }f} \< \\
            \< \sendmessageright*{y} \< \text{Check }y = f(x)
    }

    If we did not have succinct proofs, then the client would still have to run the function again to verify the output. Note this allows interactions, so this is not the go-to example.
\end{example}

\emph{Is it possible?} This remains as the large problem. Even in the na\"ive \textsf{NP} situation, we need to send the entire witness $w$ and check the entire witness.

Enter probabilistically checkable proofs (PCP):

The prover prepares a proof and the verifier will only need to check certain bits of the proof.

\begin{theorem}[PCP Theorem, Informally]
    Every \textsf{NP} language has a PCP where the verifier reads only a \emph{constant} number of bits of the proof, to gain constant soundness.
\end{theorem}

The intuition is for the prover to commit the entire proof, the verifier checks certain bits, and the prover opens commitments.


The problem with this is that the first round message is not succinct (the commitment is just as long).

\subsection{Merkle Tree Commitment Scheme}
Instead of committing linearly, we'll use a Merkle Tree, and only send the commitment/hash of the root node. We build up a binary tree where each node is the hash of its branches. Opening particular bits, the prover will send the root-to-leaf path along with siblings to prove that this opening was correct. This size will grow logarithmically with the size of the tree.

\Graphic{images/2023-03-16/merkle.png}{0.7}

We hash values in a tree format, with each parent node being the hash of its children. We only send the \emph{root note}. Whenever the verifier requests a certain bit, we send the path from the root to the bit (revealing all hashes, and siblings) to verify that this is indeed.

It's very difficult to change any bit. If we changed a bit, at some point up the path of the tree we'll have found a collision for a hash. That is to say, a specific bit being correct is predicated on whether the path to the root is valid and the root hash matches.

\emph{Can we make this hiding?} Right now, we don't guarantee the hiding property. If we only had one layer, every bit would be revealed. How can we modify this algorithm to ensure that each bit is hiding?

One solution would be to add a random string $r$ as a sibling to every leaf. However, this would require us to reveal all siblings when we're verifying a certain leaf node. We can easily modify this to \emph{salt} \emph{every} leaf node. We can add some random $r_i$ to the hash of \emph{every} bit that hides those bits.

Now, instead of sending a commitment of the entire proof, we send a Merkle Tree of the commitment of the proof. Then, when requested for certain bits $i, j, k$, we'll open those commitments as paths on the tree.

\emph{Is this zero-knowledge?} Note that in the PCP theorem, we did not have the zero-knowledge property. Our solution is that when opening commitments, we can instead provide ZK proofs for our `reveals' instead of the actual bits themselves. Asymptotically, this still preserves our succinctness property.

Theoretically, this lets us construct zk-SNARGs. In practice, there are more efficient ways to construct them, but we will not cover them now.

\subsection{Anonymous Transactions on Blockchains}

We think of the blockchain as a public ledger. Say Alice wants to send 2 Bitcoin to Bob, Alice will sign the transaction using her signing key and add that transaction onto the ledger. All transactions are public, you know which addresses sent to which addresses. The public nature of the ledger allows all parties to verify transactions.

The blockchain is maintained, in a distributed way, by many parties called "miners." Each block contains a chain of transactions;

\pseudocodeblock{
    \textbf{Alice's Account A} \< \< \textbf{Bob's Account B}\\
    \text{vk}_A \text{ (public)}, \text{sk}_A \text{ (private)} \< \sendmessageright*{2 \text{BTC (Bitcoin)}} \< \text{vk}_B \text{ (public)}, \text{sk}_B \text{ (private)}\\
    \< \< \\[][\hline]
    \textit{Transaction}\< \< \\
    \sigma = \text{Sign}_{\text{sk}_A} (\text{vk}_A, \text{vk}_B, \text{2 BTC}) \< \sendmessageright*{\sigma}\< \\
    \< \< \\[][\hline]
    \textit{Anonymous Transaction} \< \< \\
    \text{Com}(\sigma) \< \sendmessageright*{ }\< \\
    \text{NIZK: valid transaction} \< \< \\
}

There's a lot of work to make transactions anonymous. We'll hide a transaction and hide it, and use a NIZK to prove that it is a valid transaction. We want these proofs to be non-interactive and succinct (we don't want users to spend too long doing verification). This is a major application of SNARK and zk-SNARK

\subsubsection{Byzantine Agreement}
Imagine a network where we have $n$ nodes and $t$ faulty nodes. How can we get them to agree? Byzantine Fault Tolerance (BFT) Protocol states that if $n \ge 3t + 1$ it is possible to reach concensus. However, we must assume $t < n/3$ - if we do, we can agree that the next block on the blockchain is a valid block. 

In a permissionless protocol like Bitcoin, this is an issue - since anyone can become a node (i.e., become a miner), how do we ensure an honest majority, where an adversary cannot overrun the system with fake nodes? Bitcoin solves this problem with a concept called \textbf{proof of work}. Every node must use computation power to contribute to the blockchain.

In practice, the way this works is as follows: in order to find a block, the block (with some nonce) and the previous hash is hashed. Whichever node finds a block which can cash to a value with more than 30 leading zeros 'wins', and that block becomes the next one. This is a problem of raw computation and luck.

Why would miners want to devote massive amount of computation towards a game of luck? For Bitcoin, every transaction has some transaction fee. When a miner includes a transaction in a block that it solves, it gets that transaction fee. Furthermore, every block generates some new coin, which the winning miner also gets. In practice, miners often pool their computation into mining pools, which distibute the risk and reward across the mining pool.

\Graphic{images/2025/byzantine.png}{0.5}

\subsubsection{Longest Chain Rule}
It is possible for two miners to compute two valid blocks at roughly the same time, which can cause nodes to get out of sync. Bitcoin solves this issue with the longest chain rule, which states that a node should always use the longest chain. Assuming honest majority of compute, the longest chain is always valid.

In practice, a block should not be considered valid until a node is confident it is on the longest chain. This can be assumed to be 6+ blocks deep.

\Graphic{images/2025/longest-chain.png}{0.7}

\subsubsection{Extensions to Blockchain}
The protocol described above is relevant to Bitcoin. Other protocols are built differently, with upsides and downsides to each. Other topics you can research (if interested) include:
\begin{enumerate}
    \item Proof of Stake
    \item Anonymous Transactions
    \item Smart Contracts
    \item Public bulletin
\end{enumerate}