%!TEX root = ../notes.tex
\section{February 16, 2023}
\label{20230216}

We've now learned all the cryptographic primitives we need in this course.

Recall that we had a way for Alice and Bob to communicate securely, first exchanging a shared Diffie-Hellman key and then performing AES encryption.

\Graphic{images/2023-02-16/alice_and_bob.png}{0.8}

However, this is prone to a man-in-the-middle attack. One way to solve this is for parties to \emph{sign} their Diffie-Hellman publics, and verify using a publicly known verification key.

\Graphic{images/2023-02-16/signed_alice_and_bob.png}{0.8}

\emph{Is this now secure against an adversary in the middle?} Yes, because the public values are guaranteed (via our digital signature scheme) by Alice and Bob's signing key. The man in the middle does not have access to the signing key, and cannot sign a phony public value.

\subsection{SSH}

This is \emph{exactly} how the SSH algorithm works.

\Graphic{images/2023-02-16/ssh_protocol.png}{0.8}

Let's work through the steps of GitHub's SSH setup to see how it works.

\Graphic{images/2023-02-16/ssh_instruction.png}{0.8}

The instructions are given for the EDDSA-25519 algorithm, which relies on elliptic curves.

\begin{enumerate}
    \item We first generate a signing keypair $(vk_A, sk_A) \leftarrow \Gen(1^\lambda)$ via

          \texttt{\$ ssh-keygen -t ed25519 -C "your\_email@example.com"}

          $vk_A$ is the \texttt{id\_ed25519.pub} (the public key) $sk_A$ is \texttt{id\_ed25519} (the private key).
    \item We upload our public key to our account on GitHub. This is equivalent of communicating our $vk_A$ to GitHub.
    \item When we're connecting via SSH to GitHub for the first time, our terminal will prompt us that this is a new server with a new verification key.

          \texttt{> The authenticity of host `github.com (IP ADDRESS)' can't be established. \\
              > RSA key fingerprint is SHA256:nThbg6kXUpJWGl7E1IGOCspRomTxdCARLviKw6E5SY8. \\
              > Are you sure you want to continue connecting (yes/no)?}

          which we can verify against GitHub's known verification keys\footnote{The security of our web upload to GitHub, or GitHub's site which publishes the verification key, relies on the security of the website, likely through TLS. But you could also imagine exchanging keys in person, etc. }. This is the equivalent of receiving a $vk_B$ from GitHub.
\end{enumerate}

\subsection{One-Sided Secure Authentication}
In some circumstances, it's more difficult for a client to communicate their verification key to a server than it is for a server to do so. A server might publish their verification key, and trust that all clients are not compromised.

\Graphic{images/2023-02-16/one-sided_secure_authentication.png}{0.8}

\emph{What could an adversary potentially do?} The adversary could not pretend to be the server since they have no access to the server's signing key. The adversary \emph{can} pretend to be the client and talk to the server. The adversary could forward all messages sent to the server, and can also communicate $g^b, \sigma_b$ back to the client (it's a valid signature since it has not been modified).

At the end of this protocol, the client has Diffie-Hellman private $g^{ab}$ and the adversary and server will have $g^{eb}$ (where $g^e, e$ is a Diffie-Hellman keypair the adversary provided to the client). Whatever the client sends to the server cannot be decrypted by the adversary, since it is encrypted with $g^{ab}$, however, the server's communications \emph{could} be decrypted by the adversary.

This can be easily circumvented by requiring the server and user complete their handshake---the server could request a hash or encryption of the shared secret, and realize that they are communicating to an adversary when this cannot be forged by the man-in-the-middle.

\subsection{Public Key Infrastructure}

\emph{How can we know who has which public keys on the internet?} We can rely on a Public Key Infrastructure (PKI) to know each other's public keys.

If Bob purports to be \texttt{bob.com} and wants to prove that $vk_B$ belongs to him, Bob will send a certificate signing request (CSR) to a Certificate Authority (CA)\footnote{\emph{The higher beings that be}...this is companies like \emph{DigiCert}, \emph{Let's Encrypt}, etc.}.

The CA will sign the message $(\mathtt{bob.com}, vk_B)$ and send that signature $\sigma$ back to Bob. This verifies that the user of \texttt{bob.com} holds signing key $sk_B$ with public key $vk_B$.

\Graphic{images/2023-02-16/pki_cert.png}{0.8}

The standard of which is the X.509 certificate.

For example, when we try to access \texttt{facebook.com}, we can check that the certificate is valid\footnote{In browsers, this is represented by the lock symbol---clicking on that will allow you to verify that certificate.}

\Graphic{images/2023-02-16/fb_cert.png}{0.8}

This pivots on the fact that \emph{everyone} must know $vk$ of the certificate authority. We shift our trust from individual sites and users to the certificate authorities. Most devices have the $vk$s of trusted authorities built in.

\subsubsection{Certificate Chain}
In reality, there are several certificate authorities, and they also form \emph{chains} of certificate authorities.

\Graphic{images/2023-02-16/cert_chain.png}{0.8}

A Root CA\footnote{We mentioned earlier that CAs are built into devices. For example, \href{https://support.apple.com/en-us/HT213464}{here} is a list of all root certificates that are built-in for Apple devices. This can go wrong too! \href{https://en.wikipedia.org/wiki/Root_certificate\#Incidents_of_root_certificate_misuse}{CAs have been misused} which causes implications on the security of the internet.} with a known $(vk, sk)\Gen(1^\lambda)$ can first sign the $vk_1$ of an Intermediate CA1, producing cert $\mathsf{cert}_1 = \sigma \leftarrow \Sign_{sk}(vk_1)$.

Then, the Intermediate CA1 can sign a certificate for Intermediate CA2, but we'll have to preserve this chain. Intermediate CA1 could produce cert $\sigma_1\leftarrow\Sign_{sk_1}(vk_2)$, but how do we know that $sk_1$ is valid? So, we'll need to include $vk_1$ and $vk_1$'s signature signed by $sk$. That is,
\begin{align*}
    \mathsf{cert}_2 = & vk_1, \sigma \leftarrow \Sign_{sk}(vk_1),   \\
                      & vk_2, \sigma_2\leftarrow \Sign_{sk_1}(vk_2)
\end{align*}
Finally, Intermediate CA2 can sign Bob's verification key using their chain. Bob's certificate will contain
\begin{align*}
    \mathsf{cert}_B = & vk_1, \sigma \leftarrow \Sign_{sk}(vk_1),   \\
                      & vk_2, \sigma_2\leftarrow \Sign_{sk_1}(vk_2) \\
                      & vk_B, \sigma_B\leftarrow \Sign_{sk_2}(vk_B)
\end{align*}
\emph{How can an Intermediate CA restrict Bob's use of these certificates? What if Bob will then go on and start signing his own certificates for people?} We can concatenate information in each certificate that restricts its use. It could specify whether it is being issued to an \emph{end user}, or even additional information like validity time.

To protect against CAs that get compromised, certificates are short-lived and have set validity times. Additionally, certificate authorities can publish revocation lists that browsers check against when validating a certificate.

\subsection{Password-Based Authentication}
Sometimes, you also want to \emph{authenticate} with a server using a password. The na\"ive implementation is that a user with an ID sends a hash of the password $h = H(\mathrm{password})$ to the server. The server stores $(\mathrm{ID}, h)$.

\Graphic{images/2023-02-16/password_authentication.png}{0.8}

In this case, an adversary could launch an \emph{Online Dictionary Attack} and try a lot of passwords with the server.

If the server were to be compromised, and its database compromised, the adversary can conduct an \emph{Offline Dictionary Attack} on the database. Additionally, the adversary can precompute all hashes and check against the database.

\emph{How can we prevent this? }

\subsubsection{Salting}

\Graphic{images/2023-02-16/salting.png}{0.8}

One way of ensuring that the hashing is non-deterministic is for servers to generate a $\mathrm{salt}\sampledfrom\{0, 1\}^s$ and send it to the user. The user will hash $H(\mathrm{password}||\mathrm{salt})$ and send that to the server. The server stores a database of $(\mathrm{ID}, \mathrm{salt}, \mathrm{h})$.

When logging in, the user first sends their ID to the server, the server will send the salt back, the user hashes their password, and the hash is sent to the server for verification.

\emph{Does this allow the user to use a weak password?} Nope! The adversary can always brute-force the password.

However, there are still issues with this scheme...we'll discuss another technique next time, \emph{peppering}, that will make the adversary's life even harder.