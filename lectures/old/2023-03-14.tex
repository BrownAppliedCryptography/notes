%!TEX root = ../notes.tex
\section{March 14, 2023}
\label{20230314}
Comments are still open. There was a brief comment about questions in class---Peihan will still continue to ask questions (as she deems necessary) and students are very welcome to discuss amongst themselves.

There have been some comments about the mathematical nature of the course. Some students think it's a good balance, while some feel like it's too focused on theory. Going forward, Peihan will be taking more of an intuitive approach so that concepts make sense and are justified. Students also enjoyed having real-world applications of the theory.

\subsection{Zero-Knowledge Proofs for All \textsf{NP}}
\subsubsection{Intuition}
To give zero-knowledge proofs for all \textsf{NP}, we can just provide a ZK proof for \emph{one} \textsf{NP}-complete problem. We'll use the graph 3-coloring. The idea is that since all \textsf{NP} problems \emph{reduce} to an \textsf{NP}-complete problem, solving this gives us a feasible theoretical solution for proving all languages in \textsf{NP}.

% \Graphic{images/2023-03-14/3coloring.png}{0.7}

The language will be $L = \left\{ G : G\text{ has a 3-coloring} \right\}$, and our witness will be the precise 3-coloring of a given graph.

In a high level, the verifier will tell the prover two vertices, and the prover will reveal the colors of those two vertices. This follows from the protocols we've seen in the past where the prover will commit to something, be challenged by the verifier, and reveal their commitment to be valid.

\emph{What is the probability that the prover will be caught?} If the graph has at least one edge that has vertices of the same color, then the probability of being caught will be $\geq \frac{1}{|E|}$.

\emph{Can we amplify this soundness?} If we challenged 2 different edges, the verifier could learn \emph{more} information about the graph and the coloring. How might we do this while still maintaining zero-knowledge for the verifier? Every time, the prover can create a new graph with different colors (that is, permute the colors) in it. This ensures that the colors from one proof will be independent of the colors of another proof.

Then, if the verifier tries to challenge another edge, the colors of the vertices with be independent. Note thaht the prover will \emph{commit} to the colors, so it's not just a case of the prover proclaiming `oh it was red and blue' on every challenge\footnote{Think of this as putting pieces of paper on each vertex, and being asked to show two adjacent vertices.}.

Over $t$ iterations, this gives us probability $\left( 1 - \frac{1}{|E|} \right)^t$. Setting $t = \lambda\cdot E$, using $\left( 1 - \frac{1}{N} \right)^N\approx \frac{1}{e}$ gives $\left( 1 - \frac{1}{|E|} \right)^{\lambda\cdot E} = \left( \frac{1}{e} \right)^\lambda$.

\subsubsection{Commitment Schemes}
That was the intuition of this zero-knowledge proof. We want to hide some number of vertices while committing to the graph (we should not be able to change it to something else). This can be done using commitment schemes.

\begin{definition}[Commitment Scheme]
    A \ul{commitment scheme} is a way for a sender to commit to a message without revealing the message to the receiver.

    The sender, with message $\{0, 1\}$, will generate randomness $r\sampledfrom \{0, 1\}^\lambda$ and generate commitment $c := \mathrm{Com}(m;r)$. $c$ is sent to the receiver.

    When the sender wants to reveal, the sender will send $(m, r)$ (the randomness) to the receiver and the receiver can verify that $c = \mathrm{Com}(m;r)$.

    % \Graphic{images/2023-03-14/commitment.png}{0.7}

\end{definition}

\begin{example}[Pedersen Commitment]
    Let's say we have a cyclic group $\GG$ of order $q$ with generator $g$. We randomly sample element $h\sampledfrom \GG$.

    To commit $m$ with random $r\sampledfrom \ZZ_q$, the sender sends sends $c := \mathrm{Com}(m;r) = g^m\cdot h^r$. This commitment is akin to the second component of the ElGamal encryption. From this group element, the receiver \emph{cannot} infer any information from the message $m$ since it is indistinguishable from uniform from $\GG$. We can find $r$ values such that $c = g^0\cdot h^r$ as well as $r'$ such that $c = g^1\cdot h^{r'}$. This does us no good.

    $r$ essentially acts like a one-time pad to mask $m$.
\end{example}

\emph{What do we want from our commitment schemes?}
\begin{description}
    \item[Hiding.] The receiver will not be able to figure out what message was committed to. $\mathrm{Com}(0;r) \simeq \mathrm{Com}(1; s)$. There is \emph{perfectly hiding} which is that the distributions are the same $\mathrm{Com}(0;r) \equiv \mathrm{Com}(1; s)$. There is also \emph{computationally hiding} which is that the distributions are computationally indistinguishable $\mathrm{Com}(0;r) \overset{c}{\simeq} \mathrm{Com}(1; s)$.
    \item[Binding.] It is hard to find another $r$ such that $\mathrm{Com}(0;r) = \mathrm{Com}(1;r)$. There is \emph{perfectly binding} such that $\forall r, s$, $\mathrm{Com}(0;r) \neq \mathrm{Com}(1;s)$. There is also \emph{computationally hiding} such that any PPT sender cannot find $r,s$ such that $\mathrm{Com}(0;r) = \mathrm{Com}(1;s)$.
\end{description}

We need both these properties for the initial commitment to the colors.

What does our earlier example, the Pedersen commitment, satisfy? It is \emph{perfectly hiding}, $h^r$ is a random group element, which will act as a one-time-pad masking $g^m$.

We note that the Pedersen commitment cannot be perfectly binding if it is perfectly hiding---the sets must be disjoint for perfect binding, but then their distributions are not the same under the hiding property. In this case, the Pedersen commitment is computationally binding. Is it possible for a sender to find $r,s$ such that $g^0\cdot h^r = g^1\cdot h^s$? This is equivalent to finding $g = h^{r-s}$, which is equivalent to solving discrete log of $g$ base $h$ (which is computationally hard).

So, the Pedersen commitment satisfies perfectly hiding and computationally binding properties.

Using this commitment scheme, the prover will commit to every vertex color, then only open the ones that the verifier chooses.

\subsubsection{Protocol}
This gives us the protocol for the zero-knowledge proof for graph 3-coloring.

Out input is $G = (V,E)$, and some witness $\phi : v\to \left\{ 0, 1, 2 \right\}$. With some computationally hiding\footnote{Slightly different from above, but also doable.} and perfectly binding commitment scheme, we can enact the following protocol.

% \Graphic{images/2023-03-14/3color_protocol.png}{0.7}

This lets us prove all \textsf{NP} languages---we can do a reduction to the 3-coloring and prove it that way. In reality, this is expensive and merely a theoretical result.

\subsection{Circuit Satisfiability}
In reality, many choose another \textsf{NP}-complete language, the circuit satisfiability problem. The language considers an arbitrary boolean circuit which consists of \textsf{AND}, \textsf{XOR} gates. The input are certain values $x$ for input values, and witnesses $w$ are the rest of the wires. The satisfiability problem is whether there exists some $w$ to make the circuit evaluate to $1$. Since the input can be any boolean circuit, this is adaptable and widely used in implementation.

% \Graphic{images/2023-03-14/circuit_sat.png}{0.5}

This circuit model is considered a lot.
\begin{example}[Pre-Image of Hash Function]
    The function is $\mathrm{C}(x,w) = H(w) - x + 1$. The circuit will output $1$ on $w$ such that $H(w) = x$. $w$ here is the pre-image of $x$.

    This allows us to, say, represent SHA as a boolean circuit to prove the pre-image of a hash function.
\end{example}

The intuition of the zero-knowlege proof is similar. Let's say the prover has some input values. The prover will commit to the bit of every wire.

% \Graphic{images/2023-03-14/circuit_sat.png}{0.5}

For example, when a verifier asks to confirm a certain \textsf{XOR} gate, the prover will perform a \emph{small} zero-knowledge proof to prove that that gate was computed correctly. Composing commitments and using sigma protocols from before will allow us to gain the functionality we want.

Let's say
\begin{align*}
    c_1 = \mathrm{Com}(x) \\
    c_2 = \mathrm{Com}(y) \\
    c_3 = \mathrm{Com}(z)
\end{align*}
and $x = y\oplus z$. Using a sigma-\textsf{OR} protocol, we can prove
\[(y = 0, z = 0, x = 0)\ \mathsf{OR}\ (y = 0, z = 1, x = 1)\ \mathsf{OR}\ \cdots\]
This allows us to do ZK-proofs for circuit satisfiability.

\subsubsection{Proof Systems for Circuit Satisfiability}
We discuss the proof systems so far for circuit satisfiability.

% \Graphic{images/2023-03-14/circuit_sat_proof_systems.png}{0.5}

The na\"ive proof is to reveal witness $w$. This is not zero-knowledge, but is non-interactive. Using $\Sigma$-protocols, we have zero-knowledge but not non-interaction. Using the Fiat-Shamir heuristic, we get both zero-knowledge and non-interaction.

For the easiest \textsf{NP} proof, communication requires $O(|w|)$ complexity and the verifier verifies in $O(|c|)$ (linear in number of gates) complexity. For $\Sigma$-protocols, communication requires a commitment to each wire, which is $O(|c|\cdot \lambda)$ (needs a factor of $\lambda$ security paramter), and the verifier also verifies in $O(|c|\cdot \lambda)$. This is the same for NIZK.

\emph{Can we make this proof system more succinct?} Can we have communication and verification complexity to be \emph{sublinear} in $|c|$ and $|w|$?

\subsection{Succinct Non-Interactive Argument (SNARG)}
This brings us to succinct arguments, which are seemingly not quite possible.
\begin{definition}[Succinct Non-Interactive Arguments]
    A non-interactive proof/argument system is \ul{succinct} if
    \begin{itemize}
        \item The proof $\pi$ is of length $|\pi| = \mathrm{poly}(\lambda, \log |c|)$.
        \item The verifier runs in time $\mathrm{poly}(\lambda, |x|, \log|c|)$.
    \end{itemize}
\end{definition}
Additionally, SNARKs are Succinct Non-Interactive Arguments of Knowledge. A zk-SNARG or zk-SNARK additionally guarantees zero-knowledge property.

\emph{Why succinct proofs?} Here are some examples where we might want succinct proofs.
\begin{example}[Verifiable Computation]
    The client sends some $x$ to the server, along with function $f$. The server sends back $y = f(x)$ and a proof. The client wants to check if the computation was done correctly.

    % \Graphic{images/2023-03-14/verifiable_computation.png}{0.5}

    If we did not have succinct proofs, then the client would still have to run the function again to verify the output. Note this allows interactions, so this is not the go-to example.
\end{example}
\begin{example}[Anonymous Transactions on Blockchains]
    We think of the blockchain as a public ledger. Say Alice wants to send 2 Bitcoin to Bob, Alice will sign the transaction using her signing key and add that transaction onto the ledger. All transactions are public, you know which addresses sent to which addresses.

    % \Graphic{images/2023-03-14/anonymous_transactions_blockchains.png}{0.5}

    There's a lot of work to make transactions anonymous. We'll hide a transaction and hide it, and use a NIZK to prove that it is a valid transaction. We want these proofs to be non-interactive and succinct (we don't want users to spend too long doing verification). This is a major application of SNARK and zk-SNARK
\end{example}

\emph{Is it possible?} This remains as the large problem. Even in the na\"ive \textsf{NP} situation, we need to send the entire witness $w$ and check the entire witness.

Enter probabilistically checkable proofs (PCP):

The prover prepares a proof and the verifier will only need to check certain bits of the proof.

% \Graphic{images/2023-03-14/pcp.png}{0.5}

\begin{theorem}[PCP Theorem, Informally]
    Every \textsf{NP} language has a PCP where the verifier reads only a \emph{constant} number of bits of the proof, to gain constant soundness.
\end{theorem}

The intuition is for the prover to commit the entire proof, the verifier checks certain bits, and the prover opens commitments.

% \Graphic{images/2023-03-14/pcp_first_attempt.png}{0.5}

The problem with this is that the first round message is not succinct (the commitment is just as long).

Instead of committing linearly, we'll use a Merkle Tree, and only send the commitment/hash of the root node. We build up a binary tree where each node is the hash of its branches. Opening particular bits, the prover will send the root-to-leaf path along with siblings to prove that this opening was correct. This size will grow logarithmically with the size of the tree.

% \Graphic{images/2023-03-14/merkle_tree.png}{0.5}