%!TEX root = ../notes.tex
\section{January 30, 2026}
\label{20250129}

\subsection{Basic Number Theory, \emph{continued}}

We recall a couple more definitions.

\begin{definition}
    We first define the multiplicative group of integers modulo n as
    $$\ZZ_N^* = \{a\in [1, N-1]\mid \gcd(a, N) = 1\}$$
\end{definition}

\begin{definition}
    We define the Euler's totient function as
    $$\phi(N) = |\ZZ_N^*|$$
\end{definition}

\begin{example}
    If $N = p \cdot q$ where $p, q$ are prime, then $\phi(N) = (p-1)(q-1)$.
\end{example}

\begin{theorem}[Euler's Theorem]
    $\forall a, N$ where $\gcd(a, N) = 1$, we have that $a^{\phi(N)} \equiv 1 \mod N$.
\end{theorem}

\begin{corollary}
    If $$d \equiv e^{-1} \mod \phi(N)$$, then $$\forall a \in \ZZ_N^*, (a^d)^e \equiv a \mod N$$.
\end{corollary}

\subsection{RSA Encryption, \emph{continued}}
\recall that the RSA encryption algorithm contains 3 components:
\begin{description}
    \item[$\Gen(1^\lambda)$:] Generate two $n$-bit primes $p, q$. We compute $N = p\cdot q$ and $\phi(N) = (p-1)(q-1)$. Choose $e$ such that $gcd(e, \phi(N)) = 1$. We compute $d = e^{-1}\mod{\phi(N)}$. Our public key $pk = (N, e)$, our secret key is $sk = d$.
    \item[$\Enc_{pk}(m)$:] $c = m^e\mod{N}$.
    \item[$\Dec_{sk}(c)$:] $m = c^d\mod{N}$.
\end{description}
We have a few remaining questions:
\begin{enumerate}
    \item How do we generate 2 primes $p, q$? More specifically, how do we generate two \emph{large} primes efficiently?
    \item How do we choose such an $e$?
    \item How do we compute $d = e^{-1}\mod{\phi(N)}$?
    \item How do we efficiently compute $m^e\mod{N}$ and $c^d\mod N$?
\end{enumerate}


\begin{ques*}
  How do we resolve these issues to ensure the $\Gen$ step is efficient (polynomial time)?
\end{ques*}
\begin{enumerate}
    \item We pick an arbitrary number $p$ and check for primality efficiently (using Miller Rabin, a probabilistic primality test). We pick random numbers until they are prime. Since primes are `pretty dense' in the integers, this can be done efficiently.
    \item Since we're unsure whether coprime numbers are dense, we can just pick a small prime $e$. Guessing, like we did with $p, q$ is also valid, although not strictly necessary.
    \item We can compute $d$ using the Extended Euclidean Algorithm.
    \item We can repeatedly square (using fast power algorithm).
\end{enumerate}

\begin{ques*}
    What happens if we can factor $N$?
\end{ques*}

Then we can find $p$ and $q$ and calculate $\phi(N) = (p-1)(q-1)$, and then we can compute $d = e^{-1}\mod{\phi(N)}$. Thus, RSA relies crucially on the factoring problem being hard.
    
\begin{ques*}
    The above scheme is known as ``plain'' RSA. Are there any security issues?
\end{ques*}
\begin{itemize}
    \item It relies on factoring being difficult (this is the computational assumption). Post-quantum, Shor's Algorithm will break RSA.
    \item Recall last lecture that CPA (Chosen-Plaintext Attack) security was defined as an adversary not being able to discern between an encryption of $m_0$ and $m_1$, \emph{knowing} $m_0$ and $m_1$ in the clear.

          Eve could just encrypt $m_0$ and $m_1$ themselves using public $e$, and discern which of the plaintexts the ciphertext corresponds to. For RSA, this is a very concrete attack.

          The concrete reason is that the encryption algorithm $\Enc$ is \emph{deterministic}. If you encrypt the same message twice, it will be the same ciphertext. Since this scheme is deterministic, it fails CPA security - for example, an adversary can encrypt messages and look for matches. We really want to be sure that $m\sampledfrom \ZZ_N^\times$ (that it has enough entropy).
\end{itemize}

\begin{ques*}
    In practice, how can RSA be useful with these limitations?
\end{ques*}

As long as we pick the plaintext which is randomly sampled, security for RSA holds. There is also a more involved way of using RSA that is CPA-secure, but we will not go in detail of it.

\begin{remark*}
    In practice, we usually set length of $p$ and $q$ to be $1024$ bits, and the key length is $2048$ bits.

    Moreover, note that although exponentiation can be done in polynomial time, it's still a very expensive operation. This is why public key encryption is, in general, more expensive than symmetric key encryption.
\end{remark*}

\subsection{Intro to Group Theory}
\begin{definition}[Group]
    A \ul{group} is a set $\GG$ along with a binary operation $\circ$ with properties:
    \begin{description}
        \item[Closure.] $\forall g, h\in \GG$, $g\circ h\in \GG$.
        \item[Existence of an identity.] $\exists e\in \GG$ such that $\forall g\in \GG$, $e\circ g = g\circ e = g$.
        \item[Existence of inverse.] $\forall g\in \GG$, $\exists h\in \GG$ such that $g\circ h = h\circ g = e$. We denote the inverse of $g$ as $g^{-1}$.
        \item[Associativity.] $\forall g_1, g_2, g_3\in \GG$, $(g_1\circ g_2)\circ g_3 = g_1\circ(g_2\circ g_3)$.
    \end{description}

    \noindent\rule{\textwidth}{0.4pt}

    We say a group is additionally \emph{Abelian} if it satisfies commutativity
    \begin{description}
        \item[Commutativity.] $\forall g, h\in \GG$, $g\circ h = h \circ g$.
    \end{description}
    For a finite group, we use $|\GG|$ to denote its \emph{order}.
\end{definition}
\begin{example}
    $(\ZZ, +)$ is an Abelian group.

    We can check so: two integers sum to an integer, identity is $0$, the inverse of $a$ is $-a$, addition is associative and commutative.

    $(\ZZ, \cdot)$ is not a group. There is no inverse for 0 such the $0 \cdot h = 1$.

    $(\ZZ_N^\times, \cdot)$ is an Abelian group ($\cdot$ is multiplication mod $N$).
\end{example}

\begin{definition}[Cyclic Group]
    Let $\GG$ be a group of order $m$. We denote
    \[\langle g\rangle := \{e=g^0, g^1, g^2, \dots, g^{m-1}\}.\]
    $\GG$ is a \ul{cyclic group} if $\exists g\in \GG$ such that $\langle g\rangle = \GG$. $g$ is called a \ul{generator} of $\GG$.
\end{definition}

\begin{example*}
    $\ZZ_p^\times$ (for prime $p$) is a cyclic group of order $p-1$\framedfootnote{A proof of this extends beyond the scope of this course, but you are recommended to check out Math 1560 (Number Theory) or Math 1580 (Cryptography). You can take this on good faith. }.
    \[\ZZ_7^\times = \{3^0 = 1, 3^1, 3^2=2, 3^3 = 6, 3^4 = 5, 3^5 = 5\}.\]
\end{example*}
\begin{ques*}
    How do we find a generator?
\end{ques*}
For every element, we can continue taking powers until $g^\alpha = 1$ for some $\alpha$. We hope that $\alpha = p-1$ (the order of $g$ is the order of the group), but we know at least $\alpha \mid p-1$.

\subsection{Computational Assumptions}

We have a few assumptions we make called the Diffie-Hellman Assumptions, in order of \textbf{weakest to strongest}\footnote{If one can solve DLOG, we can solve CDH. Given CDH, we can solve DDH. This is why CDH is \emph{stronger} than DDH, and DDH is \emph{stronger} than DLOG. It's not necessarily true the other way around (similar to factoring and DSA assumptions). } assumptions.

Let $(\GG, q, g)\leftarrow \mathcal{G}(1^\lambda)$ be a cyclic group $\GG$ or order $q$ (a $\theta(\lambda)$-bit integer) with generator $g$. For integer groups, keys are usually 2048-bits. For elliptic curve groups, keys are usually 256-bits.

\begin{definition}[Discrete Lgoarithm (DLOG) Assumption]
    Let $x\sampledfrom \ZZ_q$. We compute $h = g^x$.

    Given $(\GG, q, g, h)$, it's computationally hard to find the exponent $x$ (classically).
\end{definition}

\begin{definition}[Computational Diffie-Hellman (CDH) Assumption]
    $x, y\sampledfrom \ZZ_q$, compute $h_1 = g^x$, $h_2 = g^y$.

    Given $(\GG, q, g, h_1, h_2)$, it's computationally hard to find $g^{xy}$.
\end{definition}

\begin{definition}[Decisional Diffie-Hellman (DDH) Assumption]
    $x, y, z\sampledfrom \ZZ_q$. Compute $h_1 = g^x$, $h_2 = g^y$.

    Given $(\GG, q, g, h_1, h_2)$, it's computationally hard to distinguish between $g^{xy}$ and $g^z$.
    \[(g^x, g^y, g^{xy}) \csimeq (g^x, g^y, g^z).\]
\end{definition}

\subsection{ElGamal Encryption}
The ElGamal encryption scheme involves the following:
\begin{description}
    \item[$\Gen(1^\lambda)$:] We generate a group $(\GG, q, g) \leftarrow \mathcal{G}(1^\lambda)$. We sample $x\sampledfrom \ZZ_q$, compute $h = g^x$. Our public key is $pk = (\GG, q, g, h)$, secret key $sk = x$.
    \item[$\Enc_{pk}(m)$:] We have $m\in\GG$. We randomly sample $y\sampledfrom \ZZ_q$, which helps prevent our ciphertext from being deterministic. Our ciphertext is $c = \langle g^y, h^y\cdot m\rangle$. Note that $h = g^x$, so $g^{xy}\csimeq g^z$ is a one-time pad for our message $m$.
    \item[$\Dec_{sk}(c)$:] To decrypt $c = \langle c_1, c_2\rangle$, we raise
        \begin{align*}
            c_1^x & = (g^y)^x = g^{xy}                                      \\
            m     & = \frac{g^{xy}\cdot m}{g^{xy}} = c_2\cdot (c_1^x)^{-1}.
        \end{align*}
\end{description}

Notes about ElGamal:
\begin{itemize}
    \item Our group can be reused! We can use a public group that is fixed. In fact, there are \emph{popular} groups out there used in practice. Some of these are Elliptic Curve groups which are much more efficient than integer groups. You don't need to use the details, yet you can use it! You can use any group, so long as the group satisfies the DDH assumption.
    \item Similar to RSA, this is breakable post-quantum. Given Shor's Algorithm, we can break discrete log.
\end{itemize}

\subsection{Secure Key Exchange}
Using DDH, we can construct something very important, \emph{secure key exchange}.
\begin{definition}[Secure Key Exchange]
    Alice and Bob sends messages back and forth, and at the end of the protocol, can agree on a shared key.

    An eavesdropper looking at said communications cannot figure out what shared key they came up with.
\end{definition}
\begin{theorem}
    \emph{Informally,} It's impossible to construct secure key exchange from secret-key encryption in a black-box way.
\end{theorem}

\begin{ques*}
    How do we build a key exchange from public-key encryption?
\end{ques*}
Bob generates a keypair $(pk, sk)$. Alice generates a shared key $k\sampledfrom \{0, 1\}^\lambda$, and sends $\Enc_{pk}(k)$ to Bob.

Using Diffie-Hellman, it's very easy. We have group $(\GG, q, g)\leftarrow \mathcal{G}(1^\lambda)$. Alice samples $x\sampledfrom \ZZ_q$ and sends $g^x$. Bob also samples $y\sampledfrom \ZZ_q$ and sends $g^y$. Both Alice and Bob compute $g^{xy} = (g^x)^y = (g^y)^x$.

\begin{center}
    \includegraphics[width=0.8\textwidth]{images/2023-02-02/diffie-hellman.png}
\end{center}

What happens in practice is that parties run Diffie-Hellman key exchange to agree on a shared key. Using that shared key, they run symmetric-key encryption. This gives us efficiency. Additionally, private-key encryptions don't rely on heavy assumptions on the security of protocols (such as the DDH, RSA assumptions).
