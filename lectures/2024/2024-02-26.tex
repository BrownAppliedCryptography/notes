%!TEX root = ../notes.tex
% Scribe: Sudatta Hor
\section{February 26, 2024}
\label{20240226}

\subsection{Zero-Knowledge Proofs}
Recall from last lecture that a Zero-Knowledge Proof (ZKP) is a scheme that allows a prover to prove to a verifier some knowledge that they have, without revealing that knowledge.

\emph{What is a proof?} We consider what a `proof system' is. For example, we'll have a \emph{statement} and a \emph{proof} that is a purported proof of that statement. What guarantees do we want from this proof system? If the statement is true, we should be able to prove it; and if the statement is false, we shouldn't be able to prove this. These are our guarantees of \emph{completeness} and \emph{soundness}.
\begin{description}
    \item[Completeness.] If a statement is true, there exists a proof that proves it is true.
    \item[Soundness.] If a statement is false, any proof cannot prove it is true.
\end{description}

We can think of NP languages from a proof system perspective.

\begin{example}[Graph 3-Coloring]
    Consider the \emph{Graph 3-coloring}.

    \Graphic{images/2023-02-28/3color.png}{0.3}

    We define our language
    \begin{align*}
        L   & = \{G: G\text{ has a 3-coloring}\} \\
        \intertext{and relation}
        R_L & = \{(G, \textrm{3Col})\}
    \end{align*}
    Our statement will be that $G$ has a 3-coloring. Our proof is providing such a coloring $(G, \mathrm{3Col})\in R_L$.

    This satisfies completeness and soundness. Every 3-colorable graph has a proof that is the 3-coloring itself, and if a graph doesn't have a 3-coloring, it will not have a proof.
\end{example}

We can think of NP languages as a proof system. A language $L$ is in $\mathsf{NP}$ if $\exists \text{poly-time} V$ (verifier) such that
\begin{description}
    \item[Completeness.] $\forall x\in L, \exists w$ (witness) such that $V(x, w) = 1$.
    \item[Soundness.] $\forall x\not\in L, \forall w^*, V(x, w^*) = 0$.
\end{description}
The prover will prove to the verifier that they have knowledge of witness $w$ without revealing the witness itself.

\begin{definition}[Zero-Knowledge Proof System]
    Let $(P, V)$ (for \emph{prover} and \emph{verifier}) be a pair of probabilistic poly-time (PPT) interactive machines. $(P, V)$ is a zero-knowledge proof system for a language $L$ with associated relation $R_L$ if
    \begin{description}
        \item[Completeness.] $\forall (x, w)\in R_L$, $\Pr[P(x, w) \leftrightarrow V(x) \text{ outputs }1] = 1$. That is, if there is a $x\in L$ with witness $w$, a prover will be able to prove to the verifier that they have knowledge of $w$.
        \item[Soundness.] $\forall x\not\in L$, $\forall P^*$, $\Pr[P^*(x) \leftrightarrow V(x)\text{ outputs }1]\simeq 0$. That is, for every $x$ not in the language, our prover $P^*$ will not be able to prove its validity to $V$, with negligible probability. If $P^*$ is PPT, we call the system a \emph{zero-knowledge argument}.
    \end{description}

\end{definition}

\subsubsection{Honest-Verifier Zero-Knowledge}
Honest-Verifier Zero-Knowledge (HVZK) can be thought of as security against a semi-honest verifier. In this scenario, the verifier will follows the protocol honestly.

\[\exists\PPT\ S\text{ s.t. }\forall(x, w)\in R_L, \mathrm{View}_V(P(x, w)\leftrightarrow V(x))\simeq S(x)\]

That is, that the transcript can be simulated by a simulator $S$ without interaction with the verifier, and without $w$.

\subsubsection{Zero-Knowledge (Malicious Verifier)}

Now assume that the verifier is malicious. In other words, the verifier can deviate from the protocol in attempt to extract more information than intended. Note that a malicious verifier $V^*$ is unable to learn anything about $w$.

We need an additional property that this is actually \emph{zero-knowledge}\framedfootnote{That is, the prover could just send the witness in the clear to the verifier, which satisfies completeness and soundness.}. We want to say that the verifier is unable to extract any additional information from the interaction between the verifier and prover. That is, even without the witness, a verifier might be able to `simulate' this transaction \emph{by themselves}!

    We'll say $\forall \mathrm{PPT}\ V^*, \exists \mathrm{PPT}\ S$ such that $\forall (x, w)\in R_L$,
    \[\mathrm{Output}_{V^*}[P(x, w)\leftrightarrow V^*(x)]\simeq S(x).\]
    That is to say, for everything in the language, the output transcript between the prover and verifier can be \emph{simulated} by the simulator without knowledge of the witness\framedfootnote{This is \emph{counterintuitive}, because if any PPT can simulate the proof by themselves, how do we know we're even talking to a prover that has a witness? This is subtle, but we give extra power to the simulator that they are allowed to \emph{rewind} the verifier to some previous step. If the transcript can be simulated, then surely no information is leaked from the protocol.}.
    \Graphic{images/2023-02-28/zkp_definition.png}{0.7}

\subsubsection{Proof Knowledge}

If the prover is able to prove, then they must know a witness $w$.

\[ \exists E \text{ s.t. } \forall p^*, \forall x, \Pr[E^{P^*(\cdot)}(x) \text{ outputs }w \text{ s.t. }(x, w) \in R_L] \cong \Pr[p^* \leftrightarrow V(x) \text{ outputs } 1]\]

\subsubsection{Zero-Knowledge Proof of Knowledge}

So we've built up our four properties:
\begin{itemize}
    \item Completeness: The prover can prove whenever $x\in R_L$.
    \item Soundness: For any $x$ not in $R_L$, the prover can only prove $x\in R_L$ with \emph{negligible} probability.
    \item Zero Knowledge: The verifier does not gain any additional information from the proof. That is, a simulator could have `thought up' the entire transcript in their head given the ability to rewind.
    \item Proof of Knowledge: An even stronger guarantee than soundness (this implies soundness)---a prover must have the witness in hand to be able to prove $x\in R_L$. That is, an extractor could interact with the prover (and rewind) to be able to extract the information of $w$ from the interaction.
\end{itemize}

\subsection{Example: Schnorr's Identification Protocol}\label{sec:mar7-schnorr}
\begin{remark*}
    \emph{The following is a hodgepodge of (arguably) clearer notes taken for Peihan's previous cryptography seminar, as well as lecture and slides from the current offering of Applied Cryptography. The notation might not correspond to lecture 1-to-1, but they are the same protocols albeit with different symbols.}
\end{remark*}
Let $G$ be a cyclic group of prime order $q$ with generator $g$, $h = g^a$. We wish to prove the relation
\[R_L = \{(g^\alpha, \alpha)\}_{\alpha\in \ZZ_q}\]

The language $L$ is given by
\[L = \{ h\in G: \exists a \in \mathbb{Z}_q \text{ s.t. }h = g^a\}\]

However, this is exactly the entire group, i.e. $L = G$. 

Generator $g$ is known, and the prover wishes to prove that they have the discrete log of $h$ ($\alpha$ where $g^\alpha\equiv h$).

\Graphic{images/2023-03-02/schnorr.png}{0.75}

Completeness here is clear, the prover is able to produce such $s$ if the prover has knowledge of $\alpha$.

\subsubsection{Proof of Knowledge}

We wish that
\[\exists\PPT\ E\text{ s.t. }\forall\PPT\ P^*, \forall x\in L, E^{P^*(\cdot)}(x)\text{ outputs }w\text{ s.t. }(x, w)\in R_L\]
``That there exists an extractor that by interacting to the prover $P^*$ that can extract $w$/$\alpha$''

In this case, $E$ can rewind the prover as well. We do so as follows:

We get the prover to commit to some $r$ and $A := g^r$, and pick $2$ $\sigma$'s (rewinding) such that we have
\begin{align*}
    g^s                                                         & = h^\sigma\cdot u               \\
    g^{s'}                                                      & = h^{\sigma'}\cdot u            \\
    \intertext{Given these two equations, we can}
    g^{s - s'}                                                  & = h^{\sigma-\sigma'}
    \intertext{Then we have}
    g^{(s - s')(\sigma-\sigma')^{-1}} = h\Longrightarrow \alpha & = (s - s')(\sigma-\sigma')^{-1}
\end{align*}

\Graphic{images/2023-03-02/schnorr_pok.png}{0.75}

\subsubsection{Honest-Verifier Zero-Knowledge}
Can we also construct Zero-Knowledge for this protocol?
\[\forall\PPT\ V^*, \exists\PPT\ S \text{ s.t. }\forall (x, w)\in R_L, \mathrm{Output}_{V^*}(P(x, w)\leftrightarrow V^*(x))\overset{C}{\simeq} S(x)\]

We first do this for Honest-Verifier Zero-Knowledge. This can be thought of as security against a semi-honest verifier.
\[\exists\PPT\ S\text{ s.t. }\forall(x, w)\in R_L, \mathrm{View}_V(P(x, w)\leftrightarrow V(x))\simeq S(x)\]
That is, that the transcript can be simulated by a simulator $S$ without interaction with the verifier, and without $w$.

\Graphic{images/2023-03-02/schnorr_hvzk.png}{0.75}

We construct a simulator that gives us specific values of $A, \sigma, s$ where we can satisfy the equation $g^s = h^\sigma\cdot A$. Once we have $s$ and $\sigma$, we can easily compute the $A$ desired.

We don't have to generate them in order. Fixing $s$ and sampling $\sigma\sampledfrom \ZZ_g$, we can compute $g^s\cdot h^{-\sigma} = A$.

\subsection{Example: Diffie-Hellman Tuple}

We want to prove that $h = g^a, u = g^b, v = g^{ab}$ is a Diffie-Hellman Tuple in a cyclic group $\GG$ of order $q$ and generator $g$.

Our witness is `private exponent' $b$. Our statement is that $\exists b\in \ZZ_q$ such that $u = g^b$ and $v = h^b$.

\Graphic{images/2023-02-28/dh_tuple_proof.png}{0.7}

The prover will randomly sample $r\sampledfrom \ZZ_q$ and send to the verifier $A:=g^r$ and $B:=h^r$. The verifier randomly samples \emph{challenge} $\sigma \sampledfrom \{0, 1\}$, and sends this challenge bit to the prover. The prover will respond with $s := \sigma\cdot b + r\pmod{q}$. If the challenge bit was $0$, $s = r$ and the verifier verifies $A = g^S$ and $B = h^S$. If the challenge bit was $1$, $s = b + r$ and the verifier verifies $u\cdot A = g^s$ and $v\cdot B = h^S$.

\textbf{Completeness:} If this statement is true, the prover will be able to convince the verifier since they have knowledge of $b$.

Next lecture, we will prove proof of knowledge and honest-verifier zero-knowledge.