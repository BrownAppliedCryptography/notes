%!TEX root = ../notes.tex
% Scribe: Sudatta Hor
\section{March 4, 2024}
\label{20240304}

\subsection{Anonymous Online Voting}

Recall anonymous online voting.

Say we have $n$ voters with votes $v_1, \dots, v_n \in \{ 0, 1\}$. Each voter encrypts their vote $\text{Enc}(v_1), \dots, \text{Enc}(v_n)$. Our goal is to compute the sum of these votes without having to decrypt each vote individually. Somehow, we must find $\text{Enc}(\sum v_i)$ then decrypt to find $\sum v_i$.

There are three questions.
\begin{enumerate}
    \item How do we compute $\text{Enc}(\sum v_i)$?
    \item How do we ensure each vote is 0 or 1?
    \item Who decrypts $\text{Enc}(\sum v_i)$?
\end{enumerate}

\subsection{Additively Homomorphic Encryption}

\begin{enumerate}
    \item Additively homomorphic encryption is taking two encryptions $\text{Enc}(m_1)$ and $\text{Enc}(m_2)$ and combine them to get $\text{Enc}(m_1 + m_2)$.
    \item Multiplicatively homomorphic encryption is taking two encryptions $\text{Enc}(m_1)$ and $\text{Enc}(m_2)$ and getting the product $\text{Enc}(m_1 \cdot m_2)$.
\end{enumerate}

\begin{example}[Elgalmal Encryption]
\textbf{Elgamal Encryption:} Cyclic group $G$ with generator $g$, and the public key is given by $\text{pk} = g^{\text{sk}}$. The encryption of message $m_1$ is given by $\text{Enc}_\text{pk}(m_1) = (g^{r_1}, \text{pk}^{r_2} \cdot m_1)$ and the encryption of message $m_2$ is given by $\text{Enc}_\text{pk}(m_2) = (g^{r_2}, \text{pk}^{r_2}\cdot m_2)$. If we multiply the first components together and then the second components together, we get $(g^{r_1 + r_2}, \text{pk}^{r_1 + r_2}\cdot (m_1 \cdot m_2))$ which is exactly $\text{Enc}(m_1 \cdot m_2)$. Thus, we have multiplicatively homomorphic encryption.

\textbf{Exponential Elgamal:} The encryption of message $m_1$ is given by $\text{Enc}_{\text{pk}} (m_1) = (g^{r_1}, \text{pk}^{r_1} \cdot g^{m_1})$ and the encryption of message $m_2$ is given by $\text{Enc}_{\text{pk}} (m_2) = (g^{r_2}, \text{pk}^{r_2} \cdot g^{m_2})$. If we multiply them together element-wise like before, we get $(g^{r_1 + r_2}, \text{pk}^{r_1 + r_2}\cdot g^{m_1 + m_2})$ which is exactly $\text{Enc}(m_1 + m_2)$. thus, we have additively homomorphic encryption.

How do we do decryption? Normally, we take $c_1 = g^{r_1 + r_2}$ and $c_2 = \text{pk}^{r_1 + r_2} \cdot g^{m_1 + m_2}$ and compute $c_2 / c_1 ^{\text{sk}}$. Usually this equals the plaintext, but in this scenario it equals $g^{m_1 + m_2}$.

In our anonymous online voting scenario, each vote will be 0 or 1. Thus, the summation of all the votes is at most $n$, where $n$ is the number of voters. This is a polynomial number of possibilities, so we can compute $g^m$ for each $m \in \{0, 1, \dots, n\}$ and see which one matches $g^{\sum m_i}$ to recover the summation of the votes.
\end{example}

\subsection{Threshold Encryption}

\begin{definition}[t-out-of-n threshold]
    In t-out-of-n threshold encryption, we must have t parties out of n parties come together in order to decrypt.
\end{definition}

Let $t$ parties be denoted by $p_1, \dots, p_t$. Each party $p_i$ works independently and runs a partial gem algorithm $\text{PartialGem}(1^{\lambda})$, which generates a public and secret key pair $(\text{pk}_i, \text{sk}_i)$. After everyone is done, we combine all of the public keys $\text{pk}_i$ to get one collective public key $\text{pk}$. A message is encrypted using it $\text{ct} \gets \text{Enc}_{\text{pk}}(m)$. Note that a single party cannot decrypt by themselves.

In order to decrypt, each party $p_i$ runs a partial decryption algorithm $\text{PartialDec}(\text{sk}_i, \text{ct})$ that gives a partial decryption $d_i$. Then, we combine all of the partial decryptions $d_i$ to get the plaintext $m$.

\subsubsection{Threshold Encryption: Elgamal}

Now we give an explicit construction of threshold encryption using Elgamal.

Each party $p_i$ generates a random secret key $\text{sk}_i \gets \mathbb{Z}_q$ and public key $\text{pk}_i = g^{\text{sk}_i}$. Let $\text{pk}$ be the product of all $\text{pk}_i$, which gives us $g^{\sum \text{sk}_i}$. Next we encrypt $\text{ct} = (c_1, c_2) = (g^r, \text{pk}^r \cdot g^m)$. In order to decrypt this, we need to compute $c_2 / c_1^{\text{sk}}$ where $\text{sk} = \sum \text{sk}_i$. However, any single party does not know $\text{sk}$ by themselves.

To decrypt, each party $p_i$ does a partial decryption by computing $d_i = c_1^{\text{sk}_i}$. When all parties come together, they can multiply all the partial decryptions $d_i$ which gives us $c_1^{\sum \text{sk}_i} = c_1 ^\text{sk}$. We can use this to compute $c_2 / c_1^\text{sk}$ and decrypt.

\subsection{Voting Framework}
\Graphic{images/2023-03-09/framework.png}{0.8}
We have some servers:
\begin{description}
    \item[Registrar.] For a voter to be able to vote, they register with the Registrar to obtain a certificate to vote. They get a certificate for their verification key.
    \item[Arbiters.] The arbiters will generate the threshold encryption keys. There will be $t$ arbiters and each will have their $(pk_i, sk_i)$. They all reveal $pk_i$ to the public, so that everyone can compute the full public key $pk$.
    \item[Voter.] The voter, using the public key, will encrypt $v_i \in\{0, 1\}$. The voter will sign this vote using their signing key. They will send this vote to the Tallyer.
    \item[Tallyer.] The tallyer will check that the signature is valid\footnote{This way, the registrar cannot figure out who has voted and who hasn't.}. Then, they will strip the signature and output $\Enc_{pk}(v_1), \dots, \Enc_{pk}(v_i), \dots, \Enc_{pk}(v_n)$.
\end{description}

\subsubsection{Correctness of Partial Decryption}

Given a cyclic group $G$ of order $q$ with generator $g$, we have three pieces of public information.
\begin{enumerate}
    \item The partial public keys of each party $\text{pk}_i \in G$.
    \item The ciphertext $c = (c_1, c_2)$.
    \item The partial decryption of each party $d_i$.
\end{enumerate}

The witness is the partial secret key $\text{sk}_i$ which is private to each party. The language for ZKP for partial decryption is

\[R_L = \{((c_1, pk_i, d_i), \smallunderbrace{sk_i}_{\text{witness}}): pk_i = g^{sk_i}\land d_i = c_1^{sk_i}\}\]

This is still the Diffie-Hellman tuple! $pk_i = g^{sk_i}, c_1 = g^r, d_i = g^{r\cdot sk_i}$. We can use the NIZK for Diffie-Hellman tuple as discussed in previous lectures.

\subsubsection{Correctness of Encryption}
We want voters to prove that their encryption is either of $0$ or $1$. We're in group $\GG$ with order $q$ and generator $g$. We have public key $pk\in \GG$, and ciphertext $c = (c_1, c_2)$. We're trying to prove the statement ``$c$ is an encryption of $0$ $\mathsf{OR}$ $c$ is an encryption of $1$.''

Our languages are then encryptions of $0$ and encryptions of $1$:
\begin{align*}
    R_{L_0} & = \{(\smallunderbrace{(pk, c_1, c_2)}_{x}, \smallunderbrace{r}_{\text{witness}}) : c_1 = g^r\land c_2 = pk^r\}        \\
    R_{L_1} & = \{(\smallunderbrace{(pk, c_1, c_2)}_{x}, \smallunderbrace{r}_{\text{witness}}) : c_1 = g^r\land c_2 = pk^r\cdot g\}
\end{align*}
where $r$ is our private key. Using this, we can prove that $c$ is an encryption of $0$ ($c_2 = pk^r$) or $c$ is an encryption of $1$ ($c_2 = pk^r\cdot g$).

\subsection{Proving AND/OR Statements}

For AND, our statements are $x_1, x_2$ and our witnesses are $w_1, w_2$. The language is given by

\[ R_{AND} = \{ ((x_1, x_2), (w_1, w_2)): (x_1, w_1) \in R_{L_1} \text{ AND } (x_2, w_2) \in R_{L_2}\}\]

To prove this language, we can use a ZKP for $R_{L_1}$ and a ZKP for $R_{L_2}$.

For OR, our statements are $x_1, x_2$ and our witness is $w$. The language is

\[ R_{OR} = \{ ((x_1, x_2), w): (x_1, w) \in R_{L_1} \text{ OR } (x_2, w) \in R_{L_2}\}\]

To prove this language, we cannot use a ZKP for $R_{L_1}$ and a ZKP for $R_{L_2}$. If we do, then we reveal whether $(x_1, w) \in R_{L_1}$ and whether $(x_2, w) \in R_{L_2}$, which is revealing more information than allowed.

To prove $R_{OR}$, we know that both languages $R_{L_1}$ and $R_{L_2}$ works with a sigma protocol. The prover is going to send $(A_1, B_1)$ for the first language and $(A_2, B_2)$ for the second language, pretending that both are correct. The verifier sends a challenge $\sigma \gets \mathbb{Z}_q$. The prover separates $\sigma$ into $\sigma_1$ and $\sigma_2$, and computes responds $S_1, S_2$ for $\sigma_1, \sigma_2$ respectively. Then the verifier will verify that $\sigma = \sigma_1 + \sigma_2$, as well as the responses $((A_1, B_1), \sigma_1, S_1)$ and $((A_2, B_2), \sigma_2, S_2)$.

How does the Prover compute a response for both statements? Since we are working with OR, we might not have inclusion in one of the languages. For that language, we will simulate it.
