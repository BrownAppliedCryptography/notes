How might one find a collision for function $H: \{ 0, 1 \}^*\to \{ 0, 1 \}^n$. We can try $H(x_1), H(x_2), \dots, H(x_q)$.

If $H(x_1)$ outputs a random value, ${0, 1}^n$, what is the probability of finding a collision?

If $q = 2^n + 1$, our probability is exactly $1$ (by pigeon-hole). If $q = 2$, our probability is $\frac{1}{2^\lambda}$ (we have to get it right on the first try). What $q$ do we need for a `reasonable' probability?

\begin{remark*}
    This is related to the birthday problem. If there are $q$ students in a class, assume each student's birthday is a random day $y_i\sampledfrom [365]$. What is the probability of a collision? $q = 366$ gives $1$, $q = 23$ gives around $50\%$, and $q = 70$ gives roughly $99.9\%$.

    We can apply this trick to our hash function. If $y_i\sampledfrom [N]$, then $q = N+1$ gives us $100\%$, but $q = \sqrt{N}$ gives $50\%$ probability.
\end{remark*}

Knowing this, we want $n = 2\lambda$ (output length of hash function). If $\lambda = 128$, we want $n$ to be around $256$.

\subsubsection{Random Oracle Model}
Another way to model a hash function is the \emph{Random Oracle Model}. We think of our hash function to be an oracle (in the sky) that can \emph{only} take input and a random output (and if you give it the same input twice, the same output).

\Graphic{images/2023-02-07/oracle.png}{0.5}

There are proofs that state that no hash functions can be a random oracle. There are schemes that can be secure in the random oracle model, but are not using hash functions\footnote{Some constructions don't rely on this model.}.

In reality, hash functions are \emph{about as good as}\footnote{But can never be\dots} random oracles. Thinking of our hash functions as random oracles gives us a good intuitive understanding of how hash functions can be used in our schemes.

In this model, the best thing that an attacker can do is to try inputs and query for outputs.

\subsubsection{Constructions for Hash Function}
\begin{description}
    \item[MDS.] Output length $128$-bit. Best known attack is in $2^{16}$. A collision was found in 2004.
\end{description}
And we also have Secure Hash Functions (SHA), founded by NIST.
\begin{description}
    \item[SHA-0.] Standardized in 1993. Output length is 160-bit. Best known attack is in $2^{39}$.
    \item[SHA-1.] Standardized in 1995. Output length is $160$-bit. Best known attack is in $2^{63}$, and a collision \emph{was} found in 2017.
    \item[SHA-2.] Standardized in 2001. Output length of 224, 256, 284, 512-bit. The most commonly used is SHA-256.
    \item[SHA-3.] There was a competition from 2007-2012 for new hash functions. SHA-3 was released in 2015, and has output length 224, 256, 2384, 512-bit. This is \emph{completely different} from SHA-2.
\end{description}

\begin{remark*}
    The folklore is that during a session at a cryptography conference, a mathematician, Xiaoyun Wang, presented slide-after-slide of attacks on MDS and SHA-0, astounding the audience.
\end{remark*}