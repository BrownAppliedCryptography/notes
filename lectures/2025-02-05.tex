%!TEX root = ../notes.tex
\section{February 05, 2025}
\label{20250205}

\subsection{Hash Function, \emph{continued}}
How might one find a collision for function $H: \{ 0, 1 \}^*\to \{ 0, 1 \}^n$. We can try $H(x_1), H(x_2), \dots, H(x_q)$.

If $H(x_1)$ outputs a random value, ${0, 1}^n$, what is the probability of finding a collision?

If $q = 2^n + 1$, our probability is exactly $1$ (by pigeon-hole). If $q = 2$, our probability is $\frac{1}{2^\lambda}$ (we have to get it right on the first try). What $q$ do we need for a `reasonable' probability?

\begin{remark*}
    This is related to the birthday problem. If there are $q$ students in a class, assume each student's birthday is a random day $y_i\sampledfrom [365]$. What is the probability of a collision? $q = 366$ gives $1$, $q = 23$ gives around $50\%$, and $q = 70$ gives roughly $99.9\%$.

    We can apply this trick to our hash function. If $y_i\sampledfrom [N]$, then $q = N+1$ gives us $100\%$, but $q = \sqrt{N}$ gives $50\%$ probability.
\end{remark*}

Knowing this, we want $n = 2\lambda$ (output length of hash function). If $\lambda = 128$, we want $n$ to be around $256$.

\subsection{Collision-Resistant Hash Function (CRHF)}
Recall that we defined a hash function to be a public and deterministic function for which it is computationally hard to find a collision. That is, finding two distinct strings $x, y$ such that $H(x) = H(y)$ is computationally difficult.

For the hash function, the input domain is arbitrary-length string, and the output is a fixed-length string - 256 bits. The security guarantee is Collision-Resistant Hash Function(CRHF).
\Graphic{images/2023-02-09/hash.png}{0.4}

\subsubsection{Random Oracle Model}
Another way to model a hash function is the \emph{Random Oracle Model}. We think of our hash function to be an oracle (in the sky) that can \emph{only} take input and a random output (and if you give it the same input twice, the same output).

\Graphic{images/2023-02-07/oracle.png}{0.5}

There are proofs that state that no hash functions can be a random oracle. There are schemes that can be secure in the random oracle model, but are not using hash functions\footnote{Some constructions don't rely on this model.}.

In reality, hash functions are \emph{about as good as}\footnote{But can never be\dots} random oracles. Thinking of our hash functions as random oracles gives us a good intuitive understanding of how hash functions can be used in our schemes.

In this model, the best thing that an attacker can do is to try inputs and query for outputs.

If you are given an arbitrary output, it's extremely hard to find its input. But it is not the case for CRHF, since CRHF only assumes you can not find a collision, thus this model is tronger than CRHF.

\Graphic{images/2023-02-09/hash-oracle.png}{0.4}

\subsubsection{Constructions for Hash Function}
\begin{description}
    \item[MD5.] Output length $128$-bit. Best known attack is in $2^{16}$. A collision was found in 2004.
\end{description}
And we also have Secure Hash Functions (SHA), founded by NIST.
\begin{description}
    \item[SHA-0.] Standardized in 1993. Output length is 160-bit. Best known attack is in $2^{39}$.
    \item[SHA-1.] Standardized in 1995. Output length is $160$-bit. Best known attack is in $2^{63}$, and a collision \emph{was} found in 2017.
    \item[SHA-2.] Standardized in 2001. Output length of 224, 256, 284, 512-bit. The most commonly used is SHA-256.
    \item[SHA-3.] There was a competition from 2007-2012 for new hash functions. SHA-3 was released in 2015, and has output length 224, 256, 2384, 512-bit. This is \emph{completely different} from SHA-2.
\end{description}

\begin{remark*}
    The folklore is that during a session at a cryptography conference, a mathematician, Xiaoyun Wang, presented slide-after-slide of attacks on MD5 and SHA-0, astounding the audience.
\end{remark*}

\subsubsection{Applications}

\begin{description}
    \item[HMAC.] We can use a hash function to conduct a MAC. Computing a tag involves computing the hash function on the key appended to the message ($k||m$). It is computationally difficult to find another $k || m'$ that produces the same hash. This is a scheme that looks like
        \[\Mac_k(m) = H(k||m).\]

        However, an adversary could potentially attach some additional $s$ to $m$ to produce $m' = m||s$ such that they can easily compute $\mathsf{tag}' = H(\mathsf{tag}||s)$. This is due to the Merkle-Damgård construction of SHA-2, which associatively tags blocks of the message one-by-one.

        \Graphic{images/2023-02-09/md.png}{0.4}

        Therefore, in practice, we use a nested MAC like
        \[\Mac_k(m) = H(k||H(k||m))\]
        and just to be sure (that we're not reusing the key), we produce $k_1, k_2$ as such
        \[\mathsf{HMAC}_k(m) = H(k_1||H(k_2||m))\]
        such that $k_1 = k\oplus \mathsf{opad}$ and $k_2 = k\oplus\mathsf{ipad}$, some one-time pads.

    \item[Hash-and-Sign.] There are some other applications of a hash function. We've seen before with RSA that we want to Hash-and-Sign, removing any homomorphism that an adversary could exploit. Additionally, this allows us to sign larger messages since they are constant size after hashing.

    \item[Password Authentication.] Another application is password authentication. Instead of storing plaintext passwords on servers, websites can store a hash of the password instead. This means that the passwords are not compromised even if the server is compromised.

    \item[Deduplicate Files.] We can also use hash functions to deduplicate files. We can hash two files to produce identifiers $h_1$ and $h_2$. If $h_1\neq h_2$, this implies $D_1\neq D_2$. If $h_1 = h_2$, it almost always\footnote{If they are not equal, we've found a collision for our hash function, which is extremely unlikely.} implies that $D_1 = D_2$.

    \item[HKDF (Key Derivation Function).] We can derive more keys from a shared key, essentially using a hash function as a pseudorandom generator (PRG).

        For example, if there is $g^{ab}$ shared key, we can do
        \[\mathsf{HMAC}(g^{ab}, \textsf{salt})\]

        Using a random seed, and concatenating a public deterministic salt $G$, we can generate a random\footnote{Computationally random, because if our computational power were to be unbounded, we can try all strings.} string.

        \Graphic{images/2023-02-09/hkdf.png}{0.8}

    \item[Pseudorandom Generator (PRG).] Given a hash function $H$, we can generate a PRG easily for any length string by generating
        \begin{align*}
            \mathsf{seed}\sampledfrom \{0, 1\}^\lambda \longrightarrow & H(\mathsf{seed}||00\cdots 00) \\
                                                                       & H(\mathsf{seed}||00\cdots 01) \\
                                                                       & H(\mathsf{seed}||00\cdots 10) \\
                                                                       & \qquad\qquad \vdots
        \end{align*}

        We can take a bit of randomness (like the way we move our mouse, type keyboard, system properties) and generate our seed.

    \item[Fast Membership Proof (Merkle Tree).] Using hash functions, we can generate Merkle Trees to prove membership. In blockchains, this is equivalent to checking if a transaction occurred.

    \item[SKE Scheme?] Could we use this to encrypt? If we have a secret key $k\sampledfrom\{0, 1\}^\lambda$, can we just encrypt by
        \[\Enc_k(m) = H(k||m)\]
        Well, we can't decrypt for one without having unbounded computational power. If our plaintext $m$ comes from a small set, like $\{0, \dots, 10\}$, we could decrypt properly. However, this is not CPA-secure, since the adversary could just query for all the messages.

        \begin{remark}
            In general, all deterministic encryption schemes are not CPA-secure.
        \end{remark}
\end{description}

\subsection{Putting it Together: Secure Communication}
This is essentially what we want to do in the second project.

We use Diffie-Hellman Key Exchange between Alice and Bob to get shared $g^{ab}$. Hashing the shared key using an HKDF, we can get shared key $k = (k_1, k_2)$ (one for AES encryption, one for HMAC). Then, they perform authenticated encryption, namely Encrypt-then-MAC.

\begin{ques*}
    Are there any issues with this scheme?
\end{ques*}
An Eve could pretend to be Alice to Bob and Bob to Alice, fudging up their shared keys. This is called a \emph{Man-in-the-Middle} attack.

\subsubsection{Diffie-Hellman Ratchet}
What if a secret key gets leaked, or cracked? One simple way to fix this is to perform a Diffie-Hellman key exchange on every message. However, this incurs additionall communications costs.

Here's another idea: with every new message (when the direction of communications shifts), the party sending the message sends a new Diffie-Hellman public key for themselves. For example, if Bob is sending a message to Alice and he knows Alice's public key $g^{a_1}$ and his previous secret was $b_1$ (hence shared $g^{a_1b_1})$, Bob will generate new key $b_2, g^{b_2}$ and encrypt using $g^{a_1b_2}$, sending $g^{b_2}$ as public to Alice. Alice can recompute the shared key before decrypting.

\Graphic{images/2023-02-09/dh-ratchet.png}{0.7}

This is the protocol used in the Signal messaging app, and is what you will implement for Project 1.

\begin{ques*}
    What if $k_1$ is leaked?
\end{ques*}
We might have leaked one key, but the other keys are still computationally hard to compute. $k_1=g^{a_1b_1}$ is known, but it's equivalent to DDH to compute $g^{a_1b_2}$ or other keys.

\begin{ques*}
    What if $b_1$ is leaked?
\end{ques*}
We can compute key $k_1=g^{a_1b_1}$ and $k_2=g^{a_2b_1}$, but no further keys are leaked, and the next round of communications (after Bob refreshes his private key $b_2$) is still secure.

\subsection{Block Cipher}
To summarize, here's what we've seen so far (this table should be familiar):
\begin{center}
    \begin{tabular}{@{}lll@{}}
        \toprule
                                      & \textbf{Symmetric-Key}                                                                               & \textbf{Public-Key}                                                                   \\ \midrule
        \textbf{Message Secrecy}      & \begin{tabular}[c]{@{}l@{}}Primitive: SKE\\ Construction: \boxed{\textbf{Block Cipher}}\end{tabular} & \begin{tabular}[c]{@{}l@{}}Primitive: PKE\\ Constructions: RSA/ElGamal\end{tabular}   \\ \midrule
        \textbf{Message Integrity}    & \begin{tabular}[c]{@{}l@{}}Primitive: MAC\\ Constructions: CBC-MAC/HMAC\end{tabular}                 & \begin{tabular}[c]{@{}l@{}}Primitive: Signature\\ Constructions: RSA/DSA\end{tabular} \\ \midrule
        \textbf{Secrecy \& Integrity} & \begin{tabular}[c]{@{}l@{}}Primitive: AE\\ Construction: Encrypt-then-MAC\end{tabular}               &                                                                                       \\ \midrule
        \textbf{Key Exchange}         &                                                                                                      & Construction: Diffie-Hellman                                                          \\ \midrule
        \textbf{Important Tool}       & \begin{tabular}[c]{@{}l@{}}Primitive: Hash function\\ Construction: SHA\end{tabular}                 &                                                                                       \\ \bottomrule
    \end{tabular}
\end{center}
The only thing we haven't seen thus far is a block cipher. We first start with the definitions.

We saw earlier that a Pseudorandom Generator (PRG) produces a string that looks random. We also have Pseudorandom Functions (PRF), which are `random-looking' functions.

\subsubsection{Pseudorandom Function (PRF)}

Our Pseudorandom Function $F$ is a keyed function\footnote{In deterministic polynomial-time.} $F : \{0, 1\}^\lambda\times \{0, 1\}^n\to\{0, 1\}^m$, $F$ will take key $k$ and input $x$ to produce output $y$, $F(k, x) = y$.

Without knowing our key $k$, $F_k$ is computationally indistinguishable from some random $f\sampledfrom\{F\mid F\{0, 1\}^n\to \{0, 1\}^m\}$.

\Graphic{images/2023-02-09/prf.png}{0.8}

We have $2^\lambda$ possible $F_k$'s, and we have $(2^m)^{2^n}$ possible functions $f$. A computationally unbounded adversary could try all possible functions and distinguish our function, since $F_k$ lives in a subset of the space of $f$. However, in reality, we can assume that $F_k$ is computationally indistinguishable from any generic function.

\subsubsection{Pseudorandom Permutation (PRP)}

A further assumption is that our function is a bijection. $F_k$ is a keyed function from $F_k : \{0, 1\}^n\to\{0, 1\}^n$. We still have $2^\lambda$ possible $F_k$'s since there are $2^\lambda$

\Graphic{images/2023-02-09/prp.png}{0.8}

\begin{ques*}
    Again, how many possible $f$'s are there?
\end{ques*}
Our first string has $2^n$ choices to map to, our second choice has $2^n - 1$, so there are
\[(2^n)(2^n-1)(2^n-2)\cdots 1 = 2^n!\]

Still, this is a much larger number than $2^\lambda$, so we still make a computational assumption that our keyed function $F_k$ is still computationally indistinguishable from a random function $f$.