%!TEX root = ../notes.tex
\section{February 9, 2023}
\label{20230209}

\emph{Logistics:} Please let us know if you have outstanding Ed or Gradescope issues!

\subsection{Hash Functions, \emph{continued}}
Recall that we defined a hash function to be a function for which it is computationally hard to find a collision. That is, finding two distinct strings $x, y$ such that $H(x) = H(y)$ is computationally difficult.

\Graphic{images/2023-02-09/hash.png}{0.4}

One model that we sometimes use to facilitate our analysis is the random oracle model. We assume our hash function is a random oracle `in the sky'that produces random hashes.

By guessing, analyzing it via the birthday problem, we require time approximately $\sqrt{n}$.

\Graphic{images/2023-02-09/hash-oracle.png}{0.4}

\subsubsection{Constructions}

The hash function constructions that are still in practice (and unbroken) are SHA-2 and SHA-3.

\subsubsection{Applications}

\begin{description}
    \item[HMAC.] We can use a hash function to conduct a MAC. Computing a tag involves computing the hash function on the key appended to the message ($k||m$). It is computationally difficult to find another $k || m'$ that produces the same hash. This is a scheme that looks like
        \[\Mac_k(m) = H(k||m).\]

        However, an adversary could potentially attach some additional $s$ to $m$ to produce $m' = m||s$ such that they can easily compute $\mathsf{tag}' = H(\mathsf{tag}||s)$. This is due to the Merkle-Damgård construction of SHA-2, which associatively tags blocks of the message one-by-one.

        \Graphic{images/2023-02-09/md.png}{0.4}

        Therefore, in practice, we use a nested MAC like
        \[\Mac_k(m) = H(k||H(k||m))\]
        and just to be sure (that we're not reusing the key), we produce $k_1, k_2$ as such
        \[\mathsf{HMAC}_k(m) = H(k_1||H(k_2||m))\]
        such that $k_1 = k\oplus \mathsf{opad}$ and $k_2 = k\oplus\mathsf{ipad}$, some one-time pads.

    \item[Hash-and-Sign.] There are some other applications of a hash function. We've seen before with RSA that we want to Hash-and-Sign, removing any homomorphism that an adversary could exploit. Additionally, this allows us to sign larger messages since they are constant size after hashing.

    \item[Password Authentication.] Another application is password authentication. Instead of storing plaintext passwords on servers, websites can store a hash of the password instead. This means that the passwords are not compromised even if the server is compromised.

    \item[Deduplicate Files.] We can also use hash functions to deduplicate files. We can hash two files to produce identifiers $h_1$ and $h_2$. If $h_1\neq h_2$, this implies $D_1\neq D_2$. If $h_1 = h_2$, it almost always\footnote{If they are not equal, we've found a collision for our hash function, which is extremely unlikely.} implies that $D_1 = D_2$.

    \item[HKDF (Key Derivation Function).] We can derive more keys from a shared key, essentially using a hash function as a pseudorandom generator (PRG).

        For example, if there is $g^{ab}$ shared key, we can do
        \[\mathsf{HMAC}(g^{ab}, \textsf{salt})\]

        Using a random seed, and adding a public deterministic salt $G$, we can generate a random\footnote{Computationally random, because if our computational power were to be unbounded, we can try all strings.} string.

        \Graphic{images/2023-02-09/hkdf.png}{0.8}

        Given a hash function $H$, we can generate a PRG easily for any length string by generating
        \begin{align*}
            \mathsf{seed}\sampledfrom \{0, 1\}^\lambda \longrightarrow & H(\mathsf{seed}||00\cdots 00) \\
                                                                       & H(\mathsf{seed}||00\cdots 01) \\
                                                                       & H(\mathsf{seed}||00\cdots 10) \\
                                                                       & \qquad\qquad \vdots
        \end{align*}

        We can take a bit of randomness (like the way we move our mouse, type keyboard, system properties) and generate our seed.

    \item[Fast Membership Proof (Merkle Tree).] Using hash functions, we can generate Merkle Trees to prove membership. In blockchains, this is equivalent to checking if a transaction occurred.

    \item[SKE Scheme?] Could we use this to encrypt? If we have a secret key $k\sampledfrom\{0, 1\}^\lambda$, can we just encrypt by
        \[\Enc_k(m) = H(k||m)\]
        Well, we can't decrypt for one without having unbounded computational power. If our plaintext $m$ comes from a small set, like $\{0, \dots, 10\}$, we could decrypt properly. However, this is not CPA-secure, since the adversary could just query for all the messages.

        \begin{remark}
            In general, all deterministic encryption schemes are not CPA-secure.
        \end{remark}
\end{description}

\subsection{Putting it Together: Secure Communication}
This is essentially what we want to do in the second project.

We use Diffie-Hellman Key Exchange between Alice and Bob to get shared $g^{ab}$. Hashing the shared key using an HKDF, we can get shraed key $k = (k_1, k_2)$ (one for AES one for HMAC). Then, they perform authenticated encryption, namely Encrypt-then-MAC.

\begin{ques*}
    Are there any issues with this scheme?
\end{ques*}
An Eve could pretend to be Alice to Bob and Bob to Alice, fudging up their shared keys. This is called a \emph{Man-in-the-Middle} attack.

\subsubsection{Diffie-Hellman Ratchet}
What if a secret key gets leaked, or cracked? One simple way to fix this is to perform a Diffie-Hellman key exchange on every message. However, this incurs additionall communications costs.

Here's another idea: with every new message (when the direction of communications shifts), the party sending the message sends a new Diffie-Hellman public key for themselves. For example, if Bob is sending a message to Alice and he knows Alice's public key $g^{a_1}$ and his previous secret was $b_1$ (hence shared $g^{a_1b_1})$, Bob will generate new key $b_2, g^{b_2}$ and encrypt using $g^{a_1b_2}$, sending $g^{b_2}$ as public to Alice. Alice can recompute the shared key before decrypting.

\Graphic{images/2023-02-09/dh-ratchet.png}{0.7}

This is the protocol used in the Signal messaging app, and is what you will implement for Project 1.

\begin{ques*}
    What if $k_1$ is leaked?
\end{ques*}
We might have leaked one key, but the other keys are still computationally hard to compute. $k_1=g^{a_1b_1}$ is known, but it's equivalent to DDH to compute $g^{a_1b_2}$ or other keys.

\begin{ques*}
    What if $b_1$ is leaked?
\end{ques*}
We can compute key $k_1=g^{a_1b_1}$ and $k_2=g^{a_2b_1}$, but no further keys are leaked, and the next round of communications (after Bob refreshes his private key $b_2$) is still secure.

\subsection{Block Cipher}
To summarize, here's what we've seen so far (this table should be familiar):
\begin{center}
    \begin{tabular}{@{}lll@{}}
        \toprule
                                      & \textbf{Symmetric-Key}                                                                               & \textbf{Public-Key}                                                                   \\ \midrule
        \textbf{Message Secrecy}      & \begin{tabular}[c]{@{}l@{}}Primitive: SKE\\ Construction: \boxed{\textbf{Block Cipher}}\end{tabular} & \begin{tabular}[c]{@{}l@{}}Primitive: PKE\\ Constructions: RSA/ElGamal\end{tabular}   \\ \midrule
        \textbf{Message Integrity}    & \begin{tabular}[c]{@{}l@{}}Primitive: MAC\\ Constructions: CBC-MAC/HMAC\end{tabular}                 & \begin{tabular}[c]{@{}l@{}}Primitive: Signature\\ Constructions: RSA/DSA\end{tabular} \\ \midrule
        \textbf{Secrecy \& Integrity} & \begin{tabular}[c]{@{}l@{}}Primitive: AE\\ Construction: Encrypt-then-MAC\end{tabular}               &                                                                                       \\ \midrule
        \textbf{Key Exchange}         &                                                                                                      & Construction: Diffie-Hellman                                                          \\ \midrule
        \textbf{Important Tool}       & \begin{tabular}[c]{@{}l@{}}Primitive: Hash function\\ Construction: SHA\end{tabular}                 &                                                                                       \\ \bottomrule
    \end{tabular}
\end{center}
The only thing we haven't seen thus far is a block cipher. We first start with the definitions.

We saw earlier that a Pseudorandom Generator (PRG) produces a string that looks random. We also have Pseudorandom Functions (PRF), which are `random-looking' functions.

\subsubsection{Pseudorandom Function (PRF)}

Our Pseudorandom Function $F$ is a keyed function\footnote{In deterministic polynomial-time.} $F : \{0, 1\}^\lambda\times \{0, 1\}^n\to\{0, 1\}^m$, $F$ will take key $k$ and input $x$ to produce output $y$, $F(k, x) = y$.

Without knowing our key $k$, $F_k$ is computationally indistinguishable from some random $f\sampledfrom\{F\mid F\{0, 1\}^n\to \{0, 1\}^m\}$.

\Graphic{images/2023-02-09/prf.png}{0.8}

We have $2^\lambda$ possible $F_k$'s, and we have $(2^m)^{2^n}$ possible functions $f$. A computationally unbounded adversary could try all possible functions and distinguish our function, since $F_k$ lives in a subset of the space of $f$. However, in reality, we can assume that $F_k$ is computationally indistinguishable from any generic function.

\subsubsection{Pseudorandom Permutation (PRP)}

A further assumption is that our function is a bijection. $F_k$ is a keyed function from $F_k : \{0, 1\}^n\to\{0, 1\}^n$. We still have $2^\lambda$ possible $F_k$'s since there are $2^\lambda$

\Graphic{images/2023-02-09/prp.png}{0.8}

\begin{ques*}
    Again, how many possible $f$'s are there?
\end{ques*}
Our first string has $2^n$ choices to map to, our second choice has $2^n - 1$, so there are
\[(2^n)(2^n-1)(2^n-2)\cdots 1 = 2^n!\]

Still, this is a much larger number than $2^\lambda$, so we still make a computational assumption that our keyed function $F_k$ is still computationally indistinguishable from a random function $f$.

\subsubsection{Block Cipher Definition}
A block cipher is a function
\[F : \{0, 1\}^\lambda\times \{0, 1\}^n\to \{0, 1\}^n\]
where $\lambda$ is the key length and $n$ is the block length. A block cipher is assumed to be a pseudorandom permutation (PRP).

Our practical construction is the Advanced Encryption Standard (AES).
\begin{itemize}
    \item Our key and block sizes are $\lambda = n = 128$.
    \item This was standardized by NIST in 2001.
    \item There was a competition from 1997-2000 to come up with a block cipher scheme. It is said that the end of the competition, competitors were simply trying to attack each other's schemes. AES was eventually selected for its efficiency.
    \item Before AES, we used the Data Encryption Standard (DES) with $\lambda = 56$ and $n = 64$. The best attack is \emph{still} a brute-force search, but key and block lengths are relatively fixed (cannot be extended).
\end{itemize}

Currently, the best attack for AES is still a brute-force search, which takes time $2^{128}$.

\subsubsection{Block Cipher Modes of Operation}
\label{sec:feb9-block-cipher-moo}
We have block cipher
\[F : \{0, 1\}^\lambda\times \{0, 1\}^\lambda\to \{0, 1\}^\lambda\]
We want to construct an SKE scheme from $F$ for arbitrary-length messages. We have some $k\sampledfrom\{0, 1\}^\lambda$. We encrypt with $\Enc_k(m)$ and decrypt $\Dec_k(c)$.

Our goal is to construct an SKE scheme that is CPA (Chosen Plaintext Attack) secure.

\textbf{Electronic Code Book (ECB) Mode:}

The easiest solution is to split up our message into blocks, and run our function $F$ on each of those blocks.

\Graphic{images/2023-02-09/ebc.png}{0.4}

However, this is not CPA secure, since each block is deterministically computed.

\textbf{Cipher Block Chaining (CBC) Mode:}

We can do something else to ensure each block's plaintext is different using an Initialization Vector (IV), sampled from $\{0, 1\}^\lambda$.

After every block, we $\mathsf{XOR}$ that block's $c_i$ with the next block's message $m_{i+1}$.

\Graphic{images/2023-02-09/cbc.png}{0.7}

\begin{ques*}
    How do we decrypt this?
\end{ques*}
We can decrypt the first block, then $\mathsf{XOR}$ $c_i$ with $F^{-1}_k(c_{i+1})$.

\begin{ques*}
    Is this secure?
\end{ques*}
Assuming $F_k$ is a valid pseudorandom permutation, this is. We'll elide the proof here.

\begin{ques*}
    Can we parallelize this? Especially the $F_k$ of $F_k^{-1}$ steps?
\end{ques*}
We can't in the case of encryption. For decryption, we can perform $F_k^{-1}$ all at once, and do all the $\mathsf{XOR}$ operations in series.

\textbf{Chained Cipher Block Chaining (Chained-CBC) Mode:}
We \emph{could} also use the previous messages' $c$ values as the IV for future messages.

\Graphic{images/2023-02-09/ccbc.png}{0.8}

\begin{ques*}
    Is this now CPA-secure?
\end{ques*}
\emph{We'll continue this next time...}