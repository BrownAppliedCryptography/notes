%!TEX root = ../notes.tex
\section{March 7, 2023}
\label{20230307}
\subsection{Zero-Knowledge Proof, \emph{continued}}
\subsubsection{Recap}
We'll review some definitions from last time, pertaining to the \emph{Zero-Knowledge Proof of Knowledge}.
\begin{description}
    \item[Completeness.] If $x$ is in the language, the prover is able to prove it.
    \item[Soundness.] If $x$ is not in the language, the probability that \emph{any} prover can prove $x\in R_L$ is negligible.
    \item[Proof of Knowledge.] Note in the soundness guarantee, if everything is in the language, we don't have the guarantee that $P$ knows the witness. This notion gives us a guarantee that the prover can only prove it if it has access to the witness. There should be an extractor that, through interacting with the prover, can extract the witness.
    \item[Honest-Verifier Zero-Knowledge.] $\exists \mathrm{PPT}\ S$ such that $\forall (x, w)\in R_L$,
        \[\mathrm{View}_V[P(x, w)\leftrightarrow V(x)]\simeq S(x).\]
        That is, a simulator can simulate the view between an honest prover and verifier.
    \item[Zero-Knoweledge.] The previous statement, but the verifier is allowed to be malicious---yet the simulator should still be able to simulate the output (with rewinding).
\end{description}

The following was review of protocols (with completeness, proof of knowledge, and HVZK/ZK properties) from last time. Refer to (Schnorr, Chaum-Pedersen, Okamoto)\todo{Link from before}.

\subsubsection{Arbitrary Linear Equations}
\begin{example}[Arbitrary Linear Equations]
    Arbitrary linear equations and do similarly as before. Say we have some private $a, b, c$ such that $u = g^{a}h^{b}$ and $v = u^av^vg^c$.
    \begin{description}
        \item[Round 1.] Sample `masks' $r_1, r_2, r_3\sampledfrom\ZZ_q$ for each $r_i$ and send the exponentiated masks. Send
            \begin{align*}
                A & = g^{r_1}h^{r_2}        \\
                B & = u^{r_1}v^{r_2}g^{r_3}
            \end{align*}
        \item[Round 2.] Public challenge $\sigma\sampledfrom\ZZ_q$.
        \item[Round 3.] Send
            \begin{align*}
                s_1 & = \sigma\cdot a + r_1 \\
                s_2 & = \sigma\cdot b + r_2 \\
                s_3 & = \sigma\cdot c + r_3
            \end{align*} for each of $a, b, c$ and verify.
    \end{description}
    % \Graphic{images/2023-03-07/ale.png}{0.75}

    Completeness, proof-of-knowledge and honest-verifier zero-knowledge done the exactly same way as we've been seeing before.
\end{example}

\subsubsection{\textsf{AND} and \textsf{OR} statements}
Let's say we have some $(x_1, w_1)\in R_{L_1}$ and $(x_2, w_2)\in R_{L_2}$, how should we prove and/or relations between these?

The case of $\mathsf{AND}$ is simple: we can just run the protocol twice.

What about an $\mathsf{OR}$ statement? We have one witness for one statement, we want to prove that \emph{at least one} is true, but without revealing to the verifier which they are proving is true exactly.

Here's what we'll do.

% \Graphic{images/2023-03-07/or.png}{0.75}

The prover will prove that \emph{both} are true by running both protocols. The prover sends $A_1, A_2$ to the verifier. The verifier responds with a single challenge $\sigma\sampledfrom \ZZ_q$. Then, the prover will provide $S_1, S_2$ based on challenges $\sigma_1, \sigma_2$, and send all this to the verifier. The verifier checks $\sigma_1 + \sigma_2 = \sigma$. The intuition behind this is that the prover can fix one $\sigma_i$, and the other is a true challenge.

We rely on the fact that if the prover knows the challenge beforehand, they can use the simulator to generate a valid proof. \textsc{wlog} say the prover knows $x_1$, then the prover will \emph{fix} $\sigma_2\sampledfrom \ZZ_q$ and generate a first round message $A_2$ knowing $\sigma_2$. Then, the prover will generate a valid first round message for the statement they know, $A_1$. When they receive challenge $\sigma$ from the verifier, they compute $\sigma_1 = \sigma - \sigma_2$ and prove $x_1$ honestly.

\emph{Completeness?} This is straightforward, it follows from the completeness of the respective protocols.

\emph{Proof of Knowledge?} How can we construct an extractor that can extract a valid witness, either $w_1$ for $x_1$ or $w_2$ for $x_2$? We'll use the previous extractors $E_1$ and $E_2$.

After challenge and response, we'll rewind to before sending challenge $\sigma$ and send a different challenge $\sigma'$. If $\sigma_1 + \sigma_2 = \sigma$, and $\sigma_1' + \sigma_2' = \sigma'\neq \sigma$. Then at least $\sigma_1\neq \sigma_1'$ or $\sigma_2\neq\sigma_2'$. Say \textsc{wlog} $\sigma_1\neq \sigma_1'$, so we know $(A_1, \sigma_1, s_1)$ and $(A_1, \sigma_1', s_1')$, and we use the corresponding extractor to extract $w_1$.

\emph{Honest Verifier Zero Knowledge?} We sample $\sigma_1, \sigma_2, S_1, S_2\sampledfrom$ randomly (then $\sigma = \sigma_1 + \sigma_2$ appears random), then compute our initial commitments $A_1, A_2$.

\subsubsection{Non-Interactive Zero-Knowledge (NIZK) Proofs}
% \Graphic{images/2023-03-07/sigma.png}{0.75}
These all fall under a class called Sigma Protocols (they look like a capital $\Sigma$). A prover will first commit to some $A$, the verifier issues a challenge $\sigma$, then the prover will provide a proof corresponding to their $A$ and $\sigma$.

What if we wanted to condense it into a protocol that was \emph{non-interactive}, and only relies on the prover sending \emph{one} round of `proof' to the verifier.

% \Graphic{images/2023-03-07/nizk.png}{0.75}

Completeness and soundness are the same. What about our previous definition of zero-knowledge? For all verifiers, will the simulator be able to simulate the one-way transcript? This is equivalent to the simulator being able to prove the statement itself\footnote{This is generally impossible! For example, a NIZK for the DH tuple that satisfies Zero-Knowledge breaks the Decisional Diffie-Hellman assumption.

    For the sake of contradiction, say we had such a simulator. To distinguish whether $(g^a, g^b, g^c)\overset{c}{\simeq}(g^a, g^b, g^{ab})$, we can feed this to the simulator to get a proof, and check with the verifier whether the proof is valid. The proof is valid if and only if it is a valid tuple. This contradicts DDH. Such a simulator had better not exist. }. Since there is only one round, the simulator `loses' its ability to rewind the prover and verifier. In the plain model, we cannot achieve a NIZK proof.

To make NIZK proofs possible, there are a few models available to us.

\begin{mdframed}[style=mdgreenbox]
    \begin{description}
        \item[Common Random String/Common Reference String (CRS):] There is a trusted third-party that both parties have access to, who generates a shared reference string.

            The power that we give to the simulator is that the simulator is allowed to generate this random/reference string together with the proof. This should be indistinguishable against the real-world.

            In reality, the CRS can be generated in a \emph{key ceremony} between parties such that no party can interfere with the generated key.
    \end{description}
    % \Graphic{images/2023-03-07/crs.png}{0.75}
\end{mdframed}

\begin{mdframed}[style=mdgreenbox]
    \begin{description}
        \item[Random Oracle (RO) Model:] The prover and verifier have access to a hash function and it is a random oracle (behaves as if it is a random function).

            The additional power we give to the simulator is that they can control the behavior of the random oracle.
    \end{description}
    % \Graphic{images/2023-03-07/rom.png}{0.75}
\end{mdframed}

\subsubsection{Fiat-Shamir Heuristic}
We can convert any Sigma protocol into a NIZK under the random oracle model. Recall that the only thing that could have gone wrong is that $\sigma\sampledfrom D$ was not computed randomly. Instead of using a challenge from the verifier, the challenge becomes $\sigma: H(x||m_1)$, a hash of the transcript so far. Since both the prover and verifier have access to the hash function, the prover can generate a challenge for themselves. After that, the prover can generate response.

A malicious verifier has no control over $\sigma$ now, so a malicious verifier cannot do anything more than seeing the proof. A malicious prover also cannot produce a valid $\sigma$ without committing to a $m_1$ first before receiving $\sigma$.

We can transform any public-coin\footnote{Every message sent from the verifier is randomly sampled from a distribution, as a challenge.} HVZK of arbitrary number of rounds into a NIZK using the Fiat-Shamir heuristic in the random oracle model.

Every public coin challenge will become the hash of the transcript so far from a random oracle. This condenses the entire proof into a single message that can be sent to the verifier (and that a verifier at a later point in time can also verify).

\begin{example}
    The Fiat-Shamir Heuristic can also transform Schnorr's Identification Protocol into Schnorr's Signature Scheme in the RO model.

    We have a cyclic group $\GG$ of order $g$, and generator $g$.

    Public verification key $vk = g^a$, secret signing key $sk = a$.

    We condense the Schnorr's Identification Protocol into a NIZK proof. To sign a message $m$, the random coin sample contains a hash of $m$ as well, \todo{Incomplete}
\end{example}

\subsection{Anonymous Voting}
\todo{intro to anonymous voting}