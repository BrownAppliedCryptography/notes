%!TEX root = ../notes.tex
\section{February 7, 2023}
\label{20230207}
\subsection{Message Integrity, \emph{reviewed}}

Last lecture, we discussed methods of authenticating a message. The symmetric-key version is called a MAC (message authentication code), the public-key version is called a digital signature. Let's review what we covered last time.

\subsubsection{Message Authentication Code}
To authenticate a message, Alice will use the private key $k$ to tag a message $m$ with a tag $t$. Bob will verify that $(m, t)$ is valid with key $k$.

\Graphic{images/2023-02-07/mac.png}{0.7}

\subsubsection{Digital Signature}
\label{sec:feb7-ds}
In the public-key version, Alice has a secret key $sk$ to sign message $m$ with signature $\sigma$. Bob (or anyone) can verify with the public key $pk$ that $(m, \sigma)$ is a valid signature.

\Graphic{images/2023-02-07/ds.png}{0.7}

\subsubsection{Syntax}
Recall the syntax of MAC and digital signatures (see \cref{sec:message-integrity:syntax}).

A message authentication code (MAC) scheme consists of $\Pi = (\Gen, \mathsf{Mac}, \mathsf{Verify})$.
\begin{description}
    \item[Generation.] $k\leftarrow \Gen(1^\lambda)$.
    \item[Authentication.] $t \leftarrow \mathsf{Mac}_k(m)$.
    \item[Verification] $0/1 := \mathsf{Verify}_k(m, t)$.
\end{description}

A digital signature scheme consists of $\Pi = (\Gen, \mathsf{Sign}, \mathsf{Verify})$.
\begin{description}
    \item[Generation.] $(sk, vk)\leftarrow \Gen(1^\lambda)$.
    \item[Authentication.] $\sigma \leftarrow \mathsf{Sign}_{sk}(m)$.
    \item[Verification] $0/1 := \mathsf{Verify}_{vk}(m, \sigma)$.
\end{description}

\subsubsection{Constructions}
We can construct MAC pratically using
\begin{itemize}
    \item Block Cipher: CBC-MAC
    \item Hash Functions: HMAC
\end{itemize}

We can construct digital signatures using
\begin{itemize}
    \item RSA signature: RSA Assumption.
    \item DSA signature: Discrete Log Assumption
    \item Lattice-Based Encryption Schemes (post-quantum secure).
\end{itemize}

\subsection{RSA Signatures}
Our RSA signatures algorithm works very similarly to RSA encryption.

We generate two $n$-bit primes $p, q$. Compute $N:=p\cdot q$ and $\phi(N) = (p-1)(q-1)$. Again choose $e$ with $\gcd(e, \phi(N)) = 1$ and invert $d = e^{-1}\mod\phi(N)$. Given $N$ and a random $y\sampledfrom\ZZ_N^\times$, it's computationally hard to find $x$ such that $x^e\equiv y\mod N$.

Similarly, $sk := d$ and $vk := (N, e)$. To sign, we compute
\[\Sign_{sk}(m) := m^d\mod{N}.\]
To verify, we compute
\[\Verify_{vk}(m, \sigma):= \sigma^e\overset{?}{\equiv}m\mod N.\]

\begin{ques*}
    Are there any security issues with RSA as we have constructed it so far?
\end{ques*}

Thinking back to our definition of chosen-message attack, if Eve knows many messages and signatures, she can compute another pair of valid message and keys. If we have messages
\begin{align*}
    m_0, \sigma_0 & = m_0^d\mod N \\
    m_1, \sigma_1 & = m_0^d\mod N
\end{align*}
We can compute $m^* := m_0\cdot m_1$ and $\sigma^* := \sigma_0 \cdot \sigma_1 = (m_0\cdot m_1)^d\mod N$.

We can do linear combinations of messages, as well as raising messages to arbitrary exponents, and we can get other messages with valid signatures.

There is an easy solution, however. We can hash our message $m$ before we sign, like so
\begin{align*}
    \Sign_{sk}(m)           & := {\color{red}H(}m{\color{red})}^d\mod{N}                         \\
    \Verify_{vk}(m, \sigma) & := \sigma^e\overset{?}{\equiv}{\color{red}H(}m{\color{red})}\mod N
\end{align*}
where $H$ is a hash function\footnote{A hash function is, briefly, a function that produces some random output that is hard to compute the inverse of.}. This is a commonly known technique called `hash-and-sign'.

\subsection{DSA Signatures}
DSA signatures are a bit more involved; they rely on the discrete log assumption and that it is hard to compute discrete logs in $\ZZ_p^\times$ and $\ZZ_q^\times$ a subgroup.

We give this definition using integer groups. There is also ECDSA which uses elliptic curve groups, which is used in cryptocurrencies\footnote{The specific curve used by Bitcoin, for example, is called \textbf{secp256k1}.}. Elliptic curves are much more efficient (especially when chosen correctly) and provide similar security guarantees.

\begin{description}
    \item[$\Gen(1^k)$:] We generate an $n$-bit\footnote{Usually, $2048$.} prime $p$ and an $m$-bit\footnote{Usually, $256$. We assume discrete log in \emph{both} of these groups.} prime $q$ such that $q\mid (p-1)$. We pick a random $\alpha\in\ZZ_p^\times$ such that $g = \alpha^{\frac{p-1}{q}}\mod{p}\neq 1$\footnote{This means that $\langle g\rangle$ has order $q$, a subgroup in $\ZZ_p^{-1}$. $g^q \equiv \alpha^{p-1}\equiv 1\mod{p}$. }

        We pick $x\sampledfrom \ZZ_q^\times$ and compute $h = g^x\mod{p}$.

        Our verification key is $vk := (p, q, g, h)$ and our signing key is $sk = x$.

    \item[$\Sign_{sk}(m)$:] Sample $y \sampledfrom \ZZ_q^\times$, compute $r = (g^y\mod{p})\mod{q}$. Compute $s = y^{-1}\cdot(H(m) + x\cdot r)\mod{q}$ and $\sigma := (r, s)$.

    \item[$\Verify_{vk}(m, \sigma)$:] We can compute
        \begin{align*}
            w   & = s^{-1}\mod{q}       \\
            u_1 & = H(m)\cdot w \mod{q} \\
            u_2 & = r\cdot w\mod{q}
        \end{align*}
        and verify $r\overset{?}{=}\left( g^{u_1}\cdot h^{u_2}\mod{p} \right)\mod{q}$.
\end{description}

\subsection{Authenticated Encryption}
Generally, Alice and Bob will first perform a Diffie-Hellman key exchange, then use that shared key to conduct Symmetric-Key Encryption.

\Graphic{images/2023-02-07/practice.png}{0.6}

In reality, we want to achieve both message secrecy and integrity \emph{at the same time}. For this, we can introduce Authenticated Encryption.

Our security definition is that our adversary can see the encryptions of many messages $m_0, m_1, m_2$. We want \textbf{CPA security}: given an encryption of either $m_0, m_1$, our adversary cannot distinguish between encryptions of the two. \emph{Additionally}, we want the property of \textbf{unforgeability}, that our adversary cannot generate a $c^*$ that is a valid encryption, such that $\Dec_{k}(c^*)\neq m_i$ for any $i$.

\Graphic{images/2023-02-07/ae.png}{0.8}

Now that we have two new primitives, we can construct Authenticated Encryption schemes.

\subsubsection{Encrypt-and-MAC?}
Given a CPA-secure SKE scheme $\Pi_1(\Gen_1, \Enc_1, \Dec_1)$ and a CMA-secure MAC scheme $\Pi_2 = (\Gen_2, \Mac_2, \Verify_2)$.

We construct an AE scheme $\Pi = (\Gen, \Enc, \Dec)$ by composing encryption and MAC. We encrypt the plaintext and also compute the MAC the \emph{plaintext}.

\Graphic{images/2023-02-07/encrypt-and-mac.png}{0.3}

\begin{description}
    \item[$\Gen(1^\lambda)$:] $k_1:=\Gen_1(1^\lambda)$, $k_2:=\Gen_2(1^\lambda)$, output $(k_1, k_2)$.

    \item[$\Enc(m)$:] To encrypt, we first encrypt ciphertext $c_1 := \Enc_1(k_1, m)$ and sign message (in plaintext) $t_2 := \Mac_2(k_2, m)$ and output $(c_1, t_2)$.

    \item[$\Dec(m)$:] We have ciphertext $c = (c_1, t_2)$. Our message is $m:=\Dec_1(k_1, c_1)$, and our verification bit $b := \Verify_1(k_2, (m, t_2))$. If $b = 1$, output $m$, otherwise we output $\bot$.
\end{description}

\begin{ques*}
    Is this scheme secure? Assuming the CPA-secure SKE scheme and CMA-secure MAC scheme, does this give us both CPA-security and unforgeability?
\end{ques*}

MAC gives you \emph{unforgeability}---it doesn't even try to hide the message at all. It's possible that the MAC scheme reveals the message in the clear. For example, we might have a MAC scheme that includes the message in the signature \emph{in the clear} (which is still secure)!

Since MAC doesn't try to hide the message. If our MAC reveals something about our message, our composed scheme $\Pi$ doesn't give us CPA-security. You might still be able to infer something about the message.

We try something else\dots

\subsubsection{Encrypt-then-MAC}
We encrypt first, then we MAC on the \emph{ciphertext}.

\Graphic{images/2023-02-07/encrypt-then-mac.png}{0.3}

\begin{description}
    \item[$\Gen(1^\lambda)$:] $k_1:=\Gen_1(1^\lambda)$, $k_2:=\Gen_2(1^\lambda)$, output $(k_1, k_2)$.

    \item[$\Enc(m)$:] To encrypt, we first encrypt ciphertext $c_1 := \Enc_1(k_1, m)$ and sign message (in ciphertext) $t_2 := \Mac_2(k_2, {\color{red} c_1})$ and output $(c_1, t_2)$.

    \item[$\Dec(m)$:] We have ciphertext $c = (c_1, t_2)$. Our message is $m:=\Dec_1(k_1, c_1)$, and our verification bit $b := \Verify_1(k_2, ({\color{red} c_1}, t_2))$. If $b = 1$, output $m$, otherwise we output $\bot$.
\end{description}

You can prove that Encrypt-then-MAC schemes are CPA-secure and unforgeable.

\subsubsection{MAC-then-Encrypt}
Similarly, we can also MAC first, encrypt the entire ciphertext and tag concatenated.

\Graphic{images/2023-02-07/mac-then-encrypt.png}{0.3}

\begin{description}
    \item[$\Gen(1^\lambda)$:] $k_1:=\Gen_1(1^\lambda)$, $k_2:=\Gen_2(1^\lambda)$, output $(k_1, k_2)$.

    \item[$\Enc(m)$:] To encrypt, we first encrypt ciphertext $c_1 := \Enc_1(k_1, m)$ and sign message (in plaintext) $t_2 := \Mac_2(k_2, {\color{red} m||t_2})$ and output $(c_1, t_2)$.

    \item[$\Dec(m)$:] We have ciphertext $c = (c_1, t_2)$. Our message is ${\color{red} m||t_2}:=\Dec_1(k_1, c_1)$, and our verification bit $b := \Verify_1(k_2, (m, t_2))$. If $b = 1$, output $m$, otherwise we output $\bot$.
\end{description}

\begin{ques*}
    Is this secure?
\end{ques*}

This doesn't satisfy a stronger security definition called Chosen Ciphertext Attack (CCA) security. We might be able to forge ciphertexts that decrypt to valid message and tags.

\subsubsection{Chosen Ciphertext Attack Security}
On top of CMA security, the adversary can now request Alice to decrypt ciphertexts $c_0, c_1, \dots$.

\Graphic{images/2023-02-07/cca.png}{0.7}

We can prove that MAC-then-Encrypt is not CCA secure.

The moral of this is that \textbf{you should always use Encrypt-then-MAC.}

\subsection{A Summary So Far}
\label{sec:feb7-summary-so-far}
To summarize, here's all we've covered so far:
\begin{center}
    \begin{tabular}{@{}lll@{}}
        \toprule
                                      & \textbf{Symmetric-KePy}                                                                & \textbf{Public-Key}                                                                   \\ \midrule
        \textbf{Message Secrecy}      & \begin{tabular}[c]{@{}l@{}}Primitive: SKE\\ Construction: Block Cipher\end{tabular}    & \begin{tabular}[c]{@{}l@{}}Primitive: PKE\\ Constructions: RSA/ElGamal\end{tabular}   \\ \midrule
        \textbf{Message Integrity}    & \begin{tabular}[c]{@{}l@{}}Primitive: MAC\\ Constructions: CBC-MAC/HMAC\end{tabular}   & \begin{tabular}[c]{@{}l@{}}Primitive: Signature\\ Constructions: RSA/DSA\end{tabular} \\ \midrule
        \textbf{Secrecy \& Integrity} & \begin{tabular}[c]{@{}l@{}}Primitive: AE\\ Construction: Encrypt-then-MAC\end{tabular} &                                                                                       \\ \midrule
        \textbf{Key Exchange}         &                                                                                        & Construction: Diffie-Hellman                                                          \\ \midrule
        \textbf{Important Tool}                & \begin{tabular}[c]{@{}l@{}}Primitive: Hash function\\ Construction: SHA\end{tabular}   &                                                                                       \\ \bottomrule
    \end{tabular}
\end{center}



\subsection{Hash Function}
A hash function is a public function
\[H: {0, 1}^*\to {0, 1}^n\]
where $n$ is order $\Theta(\lambda)$.

\Graphic{images/2023-02-07/hash.png}{0.5}

We want our hash function to be collision-resistant. That is, it's computationally hard to find $x, y\in\left\{ 0, 1 \right\}^*$ such that $x\neq y$ yet $H(x) = H(y)$ (which is called a collision).

How might one find a collision for function $H: \{ 0, 1 \}^*\to \{ 0, 1 \}^n$. We can try $H(x_1), H(x_2), \dots, H(x_q)$.

If $H(x_1)$ outputs a random value, ${0, 1}^n$, what is the probability of finding a collision?

If $q = 2^n + 1$, our probability is exactly $1$ (by pigeon-hole). If $q = 2$, our probability is $\frac{1}{2^\lambda}$ (we have to get it right on the first try). What $q$ do we need for a `reasonable' probability?

\begin{remark*}
    This is related to the birthday problem. If there are $q$ students in a class, assume each student's birthday is a random day $y_i\sampledfrom [365]$. What is the probability of a collision? $q = 366$ gives $1$, $q = 23$ gives around $50\%$, and $q = 70$ gives roughly $99.9\%$.

    We can apply this trick to our hash function. If $y_i\sampledfrom [N]$, then $q = N+1$ gives us $100\%$, but $q = \sqrt{N}$ gives $50\%$ probability.
\end{remark*}

Knowing this, we want $n = 2\lambda$ (output length of hash function). If $\lambda = 128$, we want $n$ to be around $256$.

\subsubsection{Random Oracle Model}
Another way to model a hash function is the \emph{Random Oracle Model}. We think of our hash function to be an oracle (in the sky) that can \emph{only} take input and a random output (and if you give it the same input twice, the same output).

\Graphic{images/2023-02-07/oracle.png}{0.5}

There are proofs that state that no hash functions can be a random oracle. There are schemes that can be secure in the random oracle model, but are not using hash functions\footnote{Some constructions don't rely on this model.}.

In reality, hash functions are \emph{about as good as}\footnote{But can never be\dots} random oracles. Thinking of our hash functions as random oracles gives us a good intuitive understanding of how hash functions can be used in our schemes.

In this model, the best thing that an attacker can do is to try inputs and query for outputs.

\subsubsection{Constructions for Hash Function}
\begin{description}
    \item[MDS.] Output length $128$-bit. Best known attack is in $2^{16}$. A collision was found in 2004.
\end{description}
And we also have Secure Hash Functions (SHA), founded by NIST.
\begin{description}
    \item[SHA-0.] Standardized in 1993. Output length is 160-bit. Best known attack is in $2^{39}$.
    \item[SHA-1.] Standardized in 1995. Output length is $160$-bit. Best known attack is in $2^{63}$, and a collision \emph{was} found in 2017.
    \item[SHA-2.] Standardized in 2001. Output length of 224, 256, 284, 512-bit. The most commonly used is SHA-256.
    \item[SHA-3.] There was a competition from 2007-2012 for new hash functions. SHA-3 was released in 2015, and has output length 224, 256, 2384, 512-bit. This is \emph{completely different} from SHA-2.
\end{description}

\begin{remark*}
    The folklore is that during a session at a cryptography conference, a mathematician, Xiaoyun Wang, presented slide-after-slide of attacks on MDS and SHA-0, astounding the audience.
\end{remark*}