%!TEX root = ../notes.tex
\section{April 21, 2025}
\label{20250421}

In this lecture, we will cover federated learning, differential privacy, elliptic curve cryptography.

\subsection{Federated Learning}

In Federated Learning, there is one central server and multiple users. The server wants to train and use an ML model using private data from the users. One application of this is Google mobile keyboard prediction.

This is different than PPML, and is a little bit more tricky. We cannot use secret shares because there is only one server. Furthermore, each user has data, but they might have issues participating in the protocol due to network issues and so on.

So far, we have used SGD when to look at a single data point $i$. 
$$\vec{w} \gets \vec{w} - \eta \cdot \nabla L_i(\vec{w})$$

We can look at a batch of data points, known as \textbf{batch SGD}.
$$\vec{w} \gets \vec{w} - \frac{\eta}{B} \cdot \sum_{i = 1}^B\nabla L_i(\vec{w})$$

For linear regression, the batch SGD can be written as
$$\vec{w} \gets \vec{w} - \frac{\eta}{B} \cdot X_B^T \cdot (X_B \cdot \vec{w} - Y_B)$$
where $X_B$ is a matrix with $B$ rows, where the $i$th row is $\vec{x}_i$, and likewise $Y_B$ is a column vector with $B$ entries, where the $i$th entry is $y_i$.

In federated learning, the server has the current $\vec{w}$ and wants to update it. It will select a batch of users and jointly compute the aggregation $X_B^T \cdot (X_B \cdot \vec{w} - Y_B)$. Each user can locally compute their own term, and the server will take the sum of those. This is known as \textbf{secure aggregation}. At a high level, each user computes a secret share of their local SGD, and the server collects all of these to sum them up to get the total SGD update.

This also applies to logistic regression, where the batch SGD update rule looks like the following.
$$\vec{w} \gets \vec{w} - \frac{\eta}{B} \cdot X_B^T \cdot ({\color{red}f(}X_B \cdot \vec{w}{\color{red})} - Y_B)$$

\begin{remark}
    \textit{What are some potential issues in federated learning?}
    
    \begin{itemize}
        \item A user can potentially lie about their local SGD, so it might skew the collective result. In practice, the total SGD is monitored to ensure that it is reasonable.
        \item The aggregated result can potentially reveal private information about the user. There has been some research investigating the security of federated learning. We can try to add differential privacy to improve the security, but this is hard to do since it may not be MPC friendly.
        
        What people do now is to use differential privacy only when releasing the current $\vec{w}$ to the users. To compute their local SGD, the users need to know $\vec{w}$, and thus need to know the total SGD of the previous iteration.
    \end{itemize}

    There is still a lot of work needed to be done in this area!
\end{remark}


\subsection{Differential Privacy}

Let's say that we have a database of sensitive data that we want to make public to others without compromising individual's privacy. For example, there might be a database of medical information that we want to share for medical studies.

\begin{center}
    \begin{tabular}{|c|c|c|c|c|c|c|}
        \hline
        Name & Age & Gender & Race & Weight & ZIP & Disease\\ \hline
        Alice & & & & & &\\ \hline
        Bob & & & & & &\\ \hline
        Charlie & & & & & &\\ \hline
        David & & & & & &\\ \hline
        Emily & & & & & &\\ \hline
        Fiona & & & & & &\\ \hline
    \end{tabular}
\end{center}

One attempt to simply anonymize the data by deleting personally identifiable information (PII) such as the name, date of birth, etc. This might still reveal information about the individual. For example, if you see ``Asian female, age 30, living in Providence'' you might guess it might be Peihan! Although the privacy guarantees may be weak, this is what is being used by HIPAA.

Another attempt is to only answer \textit{aggregate statistics queries}, for example only answering ``How many people in Providence have this disease?''. We can hope that this reveals less information about the individual. Unfortunately, we may still be able to infer a lot from these aggregate statistics. A similar attack actually occured involving Facebook Ads, where an attacker was able to use agregate queries to profile specific users.

Our goal is that the output shouldn't enable one to learn anything about an individual. This can be trivially achieved, for example if the output was 0 regardless of the data. Then the output does not reveal any private information of the data, but also does not provide much use. Hence, there is a tradeoff between privacy and correctness/utility.

Instead, our goal is that the output shouldn't enable one to learn \textit{much more} about an individual \textit{compared to} one with an output computed without the individual.

\begin{definition}
    For a database $D \in \mathcal{X}^n$, let $M$ be a \textit{randomized mechanism} that takes in $D$ and outputs a randomized output $M(D)$.
    $$D  \to \fbox{M} \to M(D)$$

    \textbf{$\epsilon$-Differential Privacy for M} states that for all neighboring datasets $D_1$ and $D_2$ (neighboring means that they differ in exactly one row), and for all $T \subseteq \text{range}(M)$, we have
    $$\Pr [M(D_1) \in T] \leq e^{\epsilon} \cdot \Pr[M(D_2) \in T].$$
\end{definition}

Intuitively, this states that two neighboring datasets have outputs that have similar probability distributions. If you apply $M$ on $D_1$, it will give you the red distribution, and if you apply $M$ on $D_2$, it will give you the blue distribution. Under the interval $T$, the two distributions are close.

\begin{center}
    \def\svgwidth{0.5\linewidth}
    \input{images/2024/.xdp-2024-04-22-diff-priv-pmf.pdf_tex-VFK3Hm}
\end{center}

This property can be relaxed with the following definition.

\begin{definition}
    \textbf{$\epsilon, \delta$-Differential Privacy for M} states that for all neighboring datasets $D_1$ and $D_2$, and for all $T \subseteq \text{range}(M)$, we have
    $$\Pr [M(D_1) \in T] \leq e^{\epsilon} \cdot \Pr[M(D_2) \in T] {\color{red} + \delta}.$$

    If $\epsilon$ or $\delta$ are large, then this is bad for privacy since the two distributions are allowed to differ more. The $\delta$ helps in cases where the distribution is very close to 0.
\end{definition}

An example of a randomized mechanism that is differentially private is called \textit{randomized response}.

\subsubsection{Randomized Response}

Let's say that we want to find the percentage of individuals that satisfy a predicate P. For example, the predicate could be ``is a MATH-CS concentrator''. Then the randomized response algorithm does the following.

\pseudocodeblock{
    \text{For each row }x_i:\\
    \quad 1.\ b \sample \{0, 1\}\\
    \quad 2.\ \text{If } b=0, \text{ then }y_i := P(x_i).\\
    \quad \quad \text{Otherwise, }y_i \sample \{0, 1\}\\
    M(D) := (y_1, \dots, y_n)
}

For example, if we want to find how many students are MATH-CS concentrators, for each student, we will decide to accurately report whether they are a concentrator if $b=0$. Otherwise, if $b=1$, we will just report a random value. Thus, half of the output is the actual response while the other half is just random.

\textbf{Thm:} Randomized response is $\ln 3$-differential private.

\textit{How to estimate the query output $\alpha$?} Let's say that $\alpha$ is the true percentage that satisfies $P$, and that we observe $k$ many 1's in the output. Then $k$ should be approximately
$$k \sim \frac{n}{2}\cdot\frac{1}{2} + \frac{n}{2}\cdot \alpha$$
since roughly half of individuals give random results and half of individuals give true results. Using this, we can approximate $\alpha$ from $k$.

\textit{How to make the mechanism more private?} If we want more privacy, we can adjust the distribution that $b$ is sampled from. We can adjust $b$ to be sampled as $0$ less often, thus the probability that individuals are revealing their true response is lower.

\subsubsection{Laplace Mechanism}

The Laplace mechanism is another mechanism that is popular in practice. This can compute an arbitrary real-valued function. This utilizes a notion called \textit{sensitivity} of a function.

\begin{definition}
    The \textbf{sensitivity} $\Delta f$ of a function $f: \mathcal{X}^n \to \mathbb{R}$ is defined as
    $$\Delta f := \max_{D_1 \sim D_2}|f(D_1) - f(D_2)|.$$

    $D_1\sim D_2$ means that $D_1$ and $D_2$ are neighbors.
\end{definition}

The Laplace mechanism is very simple. It justs computes $f$ and adds some noise onto it. For this reason, it is one of the most employed mechanism in practice.

\begin{definition}
    The \textbf{Laplace mechanism} is defined to be
    $$M(D) = f(D) + \text{Lap}(\Delta f / \epsilon)$$
\end{definition}

Here, Lap$(b)$ gives the Laplace distribution whose probability density function (PDF) is given by 
$$\text{PDF}(x) := \frac{1}{2b} \cdot \exp\left(-\frac{|x|}{b}\right).$$
For $x \sim \text{Lap}(b)$, we have that $\Pr [|x| \geq bt] \leq \exp (-t)$.

Below is a plot of PDF for different values of $b$.

\begin{center}
    \def\svgwidth{0.5\linewidth}
    \input{images/2024/.xdp-2024-04-22-laplace-pdf.pdf_tex-ei4hPu}
\end{center}

\textit{Is a bigger $b$ better for privacy, or worse?} A higher $b$ gives better privacy. Intuitively, the PDF becomes wider, so the $\text{Lap}(\Delta f / \epsilon)$ term becomes more random.

\begin{theorem}[Post-processing]
    If $M: X^n \to Y$ is $(\epsilon, \delta)$-differentially private, and $f:Y\to Z$ is an arbitrary randomized function, then $f\circ M : X^n \to Z$ is also $(\epsilon, \delta)$-differentially private.
\end{theorem}

\begin{theorem}[Group privacy]
    If $M: X^n \to Y$ is $(\epsilon, 0)$-differentially private, then $M$ is $(k\cdot \epsilon, 0)$-differentially private for groups of size $k$.

    In the standard definition of differential privacy, we consider two databases that differ by at most one row. However, in some cases, we may want to have a group of databases. Intuitively, you apply $M$ k times between each database, and each time you will lose $\epsilon$ privacy. In total, you lose $k \cdot \epsilon$ privacy.
\end{theorem}

\begin{theorem}[Composition]
    If $M_i: X^n \to Y$ is $(\epsilon_i, \delta_i)$-differentially private for $i = 1, \dots, k,$ then
    $$M(D) := (M_1(D), \dots, M_k(D))$$
    is $(\sum \epsilon_i, \sum \delta_i)$-differentially private.
\end{theorem}

\subsection{Elliptic Curve Cryptography}

So far, we have primarily been working with cyclic group $G$ of order $q$ with generator $g$ where DLOG/CDH/DDH holds. How large is $q$?
\begin{itemize}
    \item For integer groups, $q \sim 2048$ bits.
    \item For elliptic curve groups, $q \sim 256$ bits.
\end{itemize}
Thus, elliptic curve groups are more space-efficient for cryptography! The known attacks for elliptic curve groups scale on the order of $\sqrt{q}$, which allows us to have a smaller $q$ parameter to ensure security.

\begin{definition}
    An \textbf{elliptic curve} is the set of solutions $(x,y)$ to the equation
    $$y^2 = x^3 + ax + b$$
    where $4a^3 + 27b^2 \neq 0$. The last condition is there so that we get the elliptic curve shape.
\end{definition}

\begin{example}
    The elliptic curve to $y^2 = x^3 - x + 9$ is has a plot that looks something like the following. The solutions $(0, \pm 3), (1, \pm 3), (-1, \pm 3)$ are indicated with dots.

    \begin{center}
    \footnotesize
    \def\svgwidth{0.35\linewidth}
    \input{images/2024/.xdp-2024-04-22-elliptic-curve.pdf_tex-giJsNo}
    \end{center}
\end{example}

\textit{How to find rational points $(x,y) \in \mathbb{Q}^2$ on the curve?} There are two ways, the chord method and the tangent method.
\begin{enumerate}
    \item \textbf{Chord method:} Given a point P and a point Q on the curve, it defines a line $PQ$. This line must intersect the curve in a third point, R. For the curve in the previous example, if $P=(-1, -3)$ and $Q=(1, 3)$, then this defines a line $y=3x$. Plugging this into the equation for the curve gives us
    $$(3x)^2 = x^3 - x + 9.$$
    This is a cubic equation, so there must be 3 roots and thus 3 intersections. We know that the product of the roots must be negative the constant term, so $a_1 a_2 a_3 = -9$. We know two of the roots $a_1 = 1$ and $a_2 = -1$, so we have that $a_3 = 9$, which gives us the third point $R = (9, 27)$.

    We define this operation by $P \boxplus Q = R$. Notice that if $P, Q$ are rational points, then $R$ is rational because all of our steps to find $R$ is just multiplying, dividing, adding, and subtracting with rational numbers.

    \begin{center}
        \def\svgwidth{0.35\linewidth}
        \input{images/2024/.xdp-2024-04-22-elliptic-curve-chord.pdf_tex-JXGUox}
    \end{center}


    \pagebreak
    \item \textbf{Tangent method:} Imagine if the cubic equation had a double root. This corresponds to a point $P$ being the tangent point to the line. In this case, we apply the operation onto itself $P \boxplus P = S$.

    \begin{center}
        \def\svgwidth{0.35\linewidth}
        \input{images/2024/.xdp-2024-04-22-elliptic-curve-tangent.pdf_tex-B29VSo}
    \end{center}
\end{enumerate}
