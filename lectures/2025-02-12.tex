
% \subsubsection{CBC-MAC}
% We can use block ciphers to construct a MAC scheme. Splitting up our message into blocks, we feed blocks into $F_k$ and chain to next blocks. In the end, the final cipher output is our tag.

% \Graphic{images/2023-02-14/cbc-mac.png}{0.5}

% \emph{How do we verify?} We can just $\Mac$ the message again and check that the tag matches. If $F_k$ is invertible, we can also go the other way.

% \emph{Is this CMA secure?}
% \begin{itemize}
%     \item
%           Fixed-length messages of length $l\cdot n$? Yes, since we can only query for fixed-length messages, this gives us no additional information.
%     \item
%           Arbitrary-length messages? This is where problems arise---the adversary could first query for a message of 1 block, then 2 blocks, then 3 blocks, etc. By combining this information, they could produce new valid signatures.

%           A concrete attack is an adversary querying for $\Mac(m)$ to produce $\mathsf{tag}$, then querying for $\Mac(\mathsf{tag}) = \Mac(m||0) = \mathsf{tag}'$ which allows the adversary to forge a new message.
% \end{itemize}

% \begin{remark*}
%     Our constructions of authenticated encryption calls for an encryption scheme and MAC scheme. It's crucial that the two schemes have \emph{different keys}. Using the same key $k$ for both encryption and MAC can cuase issues (information from one could reveal something about the other).
% \end{remark*}

% We have a fix for the CMA-vulnerability in arbitrary-length messages:

% \subsubsection{Encrypt-last-block CBC-MAC (ECBC-MAC)}
% The vulnerability earlier was due to our encryption being \emph{associative}, so to speak.

% We can fix this is to use a different key for the last block:

% \Graphic{images/2023-02-14/ecbc-mac.png}{0.5}

% We could also attach length of messages to the first block, or other techniques.

% The nuance in CBC-MAC means that realistically, we almost always use HMAC.

% \ques{For CBC-MAC, if we randomly sample the IV and include it in the tag, will this be CMA secure?}

% No! Consider if the adversary queries for the tag of $m := m_1 || m_2 || m_3$ and receive the tag $t := (\text{IV}, \text{tag})$.

% The adversary can generate a new valid tag for $m^* := m_1 \oplus \text{IV} || m_2 || m_3$ and $t^* := (0^n, \text{tag})$.

% \Graphic{images/2024/2024-02-12.png}{.8}

% \subsection{Putting it Together}

% Looking back at \cref{sec:feb7-summary-so-far}, we've collected everything we need so far for secure communication.

% For Alice and Bob to communicate, they first exchange keys using a Diffie-Hellman key exchange, then perform authenticated encryption.

% \Graphic{images/2023-02-14/communicate.png}{0.7}

% However, this still does not mitigate against a man-in-the-middle attack. Thus, before exchanging keys, Alice and Bob should publish verification keys (to a digital signature scheme, see \cref{sec:feb7-ds}). Using this digital signature, Alice and Bob will each sign their Diffie-Hellman public values $g^a, g^b$ using their signing key, which will be attached to the message. They can respectively verify that these values came from each other, and not some Eve in the middle.